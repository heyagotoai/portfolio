\documentclass[11pt]{article}

    \usepackage[breakable]{tcolorbox}
    \usepackage{parskip} % Stop auto-indenting (to mimic markdown behaviour)
    

    % Basic figure setup, for now with no caption control since it's done
    % automatically by Pandoc (which extracts ![](path) syntax from Markdown).
    \usepackage{graphicx}
    % Keep aspect ratio if custom image width or height is specified
    \setkeys{Gin}{keepaspectratio}
    % Maintain compatibility with old templates. Remove in nbconvert 6.0
    \let\Oldincludegraphics\includegraphics
    % Ensure that by default, figures have no caption (until we provide a
    % proper Figure object with a Caption API and a way to capture that
    % in the conversion process - todo).
    \usepackage{caption}
    \DeclareCaptionFormat{nocaption}{}
    \captionsetup{format=nocaption,aboveskip=0pt,belowskip=0pt}

    \usepackage{float}
    \floatplacement{figure}{H} % forces figures to be placed at the correct location
    \usepackage{xcolor} % Allow colors to be defined
    \usepackage{enumerate} % Needed for markdown enumerations to work
    \usepackage{geometry} % Used to adjust the document margins
    \usepackage{amsmath} % Equations
    \usepackage{amssymb} % Equations
    \usepackage{textcomp} % defines textquotesingle
    % Hack from http://tex.stackexchange.com/a/47451/13684:
    \AtBeginDocument{%
        \def\PYZsq{\textquotesingle}% Upright quotes in Pygmentized code
    }
    \usepackage{upquote} % Upright quotes for verbatim code
    \usepackage{eurosym} % defines \euro

    \usepackage{iftex}
    \ifPDFTeX
        \usepackage[T1]{fontenc}
        \IfFileExists{alphabeta.sty}{
              \usepackage{alphabeta}
          }{
              \usepackage[mathletters]{ucs}
              \usepackage[utf8x]{inputenc}
          }
    \else
        \usepackage{fontspec}
        \usepackage{unicode-math}
    \fi

    \usepackage{fancyvrb} % verbatim replacement that allows latex
    \usepackage{grffile} % extends the file name processing of package graphics
                         % to support a larger range
    \makeatletter % fix for old versions of grffile with XeLaTeX
    \@ifpackagelater{grffile}{2019/11/01}
    {
      % Do nothing on new versions
    }
    {
      \def\Gread@@xetex#1{%
        \IfFileExists{"\Gin@base".bb}%
        {\Gread@eps{\Gin@base.bb}}%
        {\Gread@@xetex@aux#1}%
      }
    }
    \makeatother
    \usepackage[Export]{adjustbox} % Used to constrain images to a maximum size
    \adjustboxset{max size={0.9\linewidth}{0.9\paperheight}}

    % The hyperref package gives us a pdf with properly built
    % internal navigation ('pdf bookmarks' for the table of contents,
    % internal cross-reference links, web links for URLs, etc.)
    \usepackage{hyperref}
    % The default LaTeX title has an obnoxious amount of whitespace. By default,
    % titling removes some of it. It also provides customization options.
    \usepackage{titling}
    \usepackage{longtable} % longtable support required by pandoc >1.10
    \usepackage{booktabs}  % table support for pandoc > 1.12.2
    \usepackage{array}     % table support for pandoc >= 2.11.3
    \usepackage{calc}      % table minipage width calculation for pandoc >= 2.11.1
    \usepackage[inline]{enumitem} % IRkernel/repr support (it uses the enumerate* environment)
    \usepackage[normalem]{ulem} % ulem is needed to support strikethroughs (\sout)
                                % normalem makes italics be italics, not underlines
    \usepackage{soul}      % strikethrough (\st) support for pandoc >= 3.0.0
    \usepackage{mathrsfs}
    

    
    % Colors for the hyperref package
    \definecolor{urlcolor}{rgb}{0,.145,.698}
    \definecolor{linkcolor}{rgb}{.71,0.21,0.01}
    \definecolor{citecolor}{rgb}{.12,.54,.11}

    % ANSI colors
    \definecolor{ansi-black}{HTML}{3E424D}
    \definecolor{ansi-black-intense}{HTML}{282C36}
    \definecolor{ansi-red}{HTML}{E75C58}
    \definecolor{ansi-red-intense}{HTML}{B22B31}
    \definecolor{ansi-green}{HTML}{00A250}
    \definecolor{ansi-green-intense}{HTML}{007427}
    \definecolor{ansi-yellow}{HTML}{DDB62B}
    \definecolor{ansi-yellow-intense}{HTML}{B27D12}
    \definecolor{ansi-blue}{HTML}{208FFB}
    \definecolor{ansi-blue-intense}{HTML}{0065CA}
    \definecolor{ansi-magenta}{HTML}{D160C4}
    \definecolor{ansi-magenta-intense}{HTML}{A03196}
    \definecolor{ansi-cyan}{HTML}{60C6C8}
    \definecolor{ansi-cyan-intense}{HTML}{258F8F}
    \definecolor{ansi-white}{HTML}{C5C1B4}
    \definecolor{ansi-white-intense}{HTML}{A1A6B2}
    \definecolor{ansi-default-inverse-fg}{HTML}{FFFFFF}
    \definecolor{ansi-default-inverse-bg}{HTML}{000000}

    % common color for the border for error outputs.
    \definecolor{outerrorbackground}{HTML}{FFDFDF}

    % commands and environments needed by pandoc snippets
    % extracted from the output of `pandoc -s`
    \providecommand{\tightlist}{%
      \setlength{\itemsep}{0pt}\setlength{\parskip}{0pt}}
    \DefineVerbatimEnvironment{Highlighting}{Verbatim}{commandchars=\\\{\}}
    % Add ',fontsize=\small' for more characters per line
    \newenvironment{Shaded}{}{}
    \newcommand{\KeywordTok}[1]{\textcolor[rgb]{0.00,0.44,0.13}{\textbf{{#1}}}}
    \newcommand{\DataTypeTok}[1]{\textcolor[rgb]{0.56,0.13,0.00}{{#1}}}
    \newcommand{\DecValTok}[1]{\textcolor[rgb]{0.25,0.63,0.44}{{#1}}}
    \newcommand{\BaseNTok}[1]{\textcolor[rgb]{0.25,0.63,0.44}{{#1}}}
    \newcommand{\FloatTok}[1]{\textcolor[rgb]{0.25,0.63,0.44}{{#1}}}
    \newcommand{\CharTok}[1]{\textcolor[rgb]{0.25,0.44,0.63}{{#1}}}
    \newcommand{\StringTok}[1]{\textcolor[rgb]{0.25,0.44,0.63}{{#1}}}
    \newcommand{\CommentTok}[1]{\textcolor[rgb]{0.38,0.63,0.69}{\textit{{#1}}}}
    \newcommand{\OtherTok}[1]{\textcolor[rgb]{0.00,0.44,0.13}{{#1}}}
    \newcommand{\AlertTok}[1]{\textcolor[rgb]{1.00,0.00,0.00}{\textbf{{#1}}}}
    \newcommand{\FunctionTok}[1]{\textcolor[rgb]{0.02,0.16,0.49}{{#1}}}
    \newcommand{\RegionMarkerTok}[1]{{#1}}
    \newcommand{\ErrorTok}[1]{\textcolor[rgb]{1.00,0.00,0.00}{\textbf{{#1}}}}
    \newcommand{\NormalTok}[1]{{#1}}

    % Additional commands for more recent versions of Pandoc
    \newcommand{\ConstantTok}[1]{\textcolor[rgb]{0.53,0.00,0.00}{{#1}}}
    \newcommand{\SpecialCharTok}[1]{\textcolor[rgb]{0.25,0.44,0.63}{{#1}}}
    \newcommand{\VerbatimStringTok}[1]{\textcolor[rgb]{0.25,0.44,0.63}{{#1}}}
    \newcommand{\SpecialStringTok}[1]{\textcolor[rgb]{0.73,0.40,0.53}{{#1}}}
    \newcommand{\ImportTok}[1]{{#1}}
    \newcommand{\DocumentationTok}[1]{\textcolor[rgb]{0.73,0.13,0.13}{\textit{{#1}}}}
    \newcommand{\AnnotationTok}[1]{\textcolor[rgb]{0.38,0.63,0.69}{\textbf{\textit{{#1}}}}}
    \newcommand{\CommentVarTok}[1]{\textcolor[rgb]{0.38,0.63,0.69}{\textbf{\textit{{#1}}}}}
    \newcommand{\VariableTok}[1]{\textcolor[rgb]{0.10,0.09,0.49}{{#1}}}
    \newcommand{\ControlFlowTok}[1]{\textcolor[rgb]{0.00,0.44,0.13}{\textbf{{#1}}}}
    \newcommand{\OperatorTok}[1]{\textcolor[rgb]{0.40,0.40,0.40}{{#1}}}
    \newcommand{\BuiltInTok}[1]{{#1}}
    \newcommand{\ExtensionTok}[1]{{#1}}
    \newcommand{\PreprocessorTok}[1]{\textcolor[rgb]{0.74,0.48,0.00}{{#1}}}
    \newcommand{\AttributeTok}[1]{\textcolor[rgb]{0.49,0.56,0.16}{{#1}}}
    \newcommand{\InformationTok}[1]{\textcolor[rgb]{0.38,0.63,0.69}{\textbf{\textit{{#1}}}}}
    \newcommand{\WarningTok}[1]{\textcolor[rgb]{0.38,0.63,0.69}{\textbf{\textit{{#1}}}}}
    \makeatletter
    \newsavebox\pandoc@box
    \newcommand*\pandocbounded[1]{%
      \sbox\pandoc@box{#1}%
      % scaling factors for width and height
      \Gscale@div\@tempa\textheight{\dimexpr\ht\pandoc@box+\dp\pandoc@box\relax}%
      \Gscale@div\@tempb\linewidth{\wd\pandoc@box}%
      % select the smaller of both
      \ifdim\@tempb\p@<\@tempa\p@
        \let\@tempa\@tempb
      \fi
      % scaling accordingly (\@tempa < 1)
      \ifdim\@tempa\p@<\p@
        \scalebox{\@tempa}{\usebox\pandoc@box}%
      % scaling not needed, use as it is
      \else
        \usebox{\pandoc@box}%
      \fi
    }
    \makeatother

    % Define a nice break command that doesn't care if a line doesn't already
    % exist.
    \def\br{\hspace*{\fill} \\* }
    % Math Jax compatibility definitions
    \def\gt{>}
    \def\lt{<}
    \let\Oldtex\TeX
    \let\Oldlatex\LaTeX
    \renewcommand{\TeX}{\textrm{\Oldtex}}
    \renewcommand{\LaTeX}{\textrm{\Oldlatex}}
    % Document parameters
    % Document title
    \title{titanic\_prezentacja}
    
    
    
    
    
    
    
% Pygments definitions
\makeatletter
\def\PY@reset{\let\PY@it=\relax \let\PY@bf=\relax%
    \let\PY@ul=\relax \let\PY@tc=\relax%
    \let\PY@bc=\relax \let\PY@ff=\relax}
\def\PY@tok#1{\csname PY@tok@#1\endcsname}
\def\PY@toks#1+{\ifx\relax#1\empty\else%
    \PY@tok{#1}\expandafter\PY@toks\fi}
\def\PY@do#1{\PY@bc{\PY@tc{\PY@ul{%
    \PY@it{\PY@bf{\PY@ff{#1}}}}}}}
\def\PY#1#2{\PY@reset\PY@toks#1+\relax+\PY@do{#2}}

\@namedef{PY@tok@w}{\def\PY@tc##1{\textcolor[rgb]{0.73,0.73,0.73}{##1}}}
\@namedef{PY@tok@c}{\let\PY@it=\textit\def\PY@tc##1{\textcolor[rgb]{0.24,0.48,0.48}{##1}}}
\@namedef{PY@tok@cp}{\def\PY@tc##1{\textcolor[rgb]{0.61,0.40,0.00}{##1}}}
\@namedef{PY@tok@k}{\let\PY@bf=\textbf\def\PY@tc##1{\textcolor[rgb]{0.00,0.50,0.00}{##1}}}
\@namedef{PY@tok@kp}{\def\PY@tc##1{\textcolor[rgb]{0.00,0.50,0.00}{##1}}}
\@namedef{PY@tok@kt}{\def\PY@tc##1{\textcolor[rgb]{0.69,0.00,0.25}{##1}}}
\@namedef{PY@tok@o}{\def\PY@tc##1{\textcolor[rgb]{0.40,0.40,0.40}{##1}}}
\@namedef{PY@tok@ow}{\let\PY@bf=\textbf\def\PY@tc##1{\textcolor[rgb]{0.67,0.13,1.00}{##1}}}
\@namedef{PY@tok@nb}{\def\PY@tc##1{\textcolor[rgb]{0.00,0.50,0.00}{##1}}}
\@namedef{PY@tok@nf}{\def\PY@tc##1{\textcolor[rgb]{0.00,0.00,1.00}{##1}}}
\@namedef{PY@tok@nc}{\let\PY@bf=\textbf\def\PY@tc##1{\textcolor[rgb]{0.00,0.00,1.00}{##1}}}
\@namedef{PY@tok@nn}{\let\PY@bf=\textbf\def\PY@tc##1{\textcolor[rgb]{0.00,0.00,1.00}{##1}}}
\@namedef{PY@tok@ne}{\let\PY@bf=\textbf\def\PY@tc##1{\textcolor[rgb]{0.80,0.25,0.22}{##1}}}
\@namedef{PY@tok@nv}{\def\PY@tc##1{\textcolor[rgb]{0.10,0.09,0.49}{##1}}}
\@namedef{PY@tok@no}{\def\PY@tc##1{\textcolor[rgb]{0.53,0.00,0.00}{##1}}}
\@namedef{PY@tok@nl}{\def\PY@tc##1{\textcolor[rgb]{0.46,0.46,0.00}{##1}}}
\@namedef{PY@tok@ni}{\let\PY@bf=\textbf\def\PY@tc##1{\textcolor[rgb]{0.44,0.44,0.44}{##1}}}
\@namedef{PY@tok@na}{\def\PY@tc##1{\textcolor[rgb]{0.41,0.47,0.13}{##1}}}
\@namedef{PY@tok@nt}{\let\PY@bf=\textbf\def\PY@tc##1{\textcolor[rgb]{0.00,0.50,0.00}{##1}}}
\@namedef{PY@tok@nd}{\def\PY@tc##1{\textcolor[rgb]{0.67,0.13,1.00}{##1}}}
\@namedef{PY@tok@s}{\def\PY@tc##1{\textcolor[rgb]{0.73,0.13,0.13}{##1}}}
\@namedef{PY@tok@sd}{\let\PY@it=\textit\def\PY@tc##1{\textcolor[rgb]{0.73,0.13,0.13}{##1}}}
\@namedef{PY@tok@si}{\let\PY@bf=\textbf\def\PY@tc##1{\textcolor[rgb]{0.64,0.35,0.47}{##1}}}
\@namedef{PY@tok@se}{\let\PY@bf=\textbf\def\PY@tc##1{\textcolor[rgb]{0.67,0.36,0.12}{##1}}}
\@namedef{PY@tok@sr}{\def\PY@tc##1{\textcolor[rgb]{0.64,0.35,0.47}{##1}}}
\@namedef{PY@tok@ss}{\def\PY@tc##1{\textcolor[rgb]{0.10,0.09,0.49}{##1}}}
\@namedef{PY@tok@sx}{\def\PY@tc##1{\textcolor[rgb]{0.00,0.50,0.00}{##1}}}
\@namedef{PY@tok@m}{\def\PY@tc##1{\textcolor[rgb]{0.40,0.40,0.40}{##1}}}
\@namedef{PY@tok@gh}{\let\PY@bf=\textbf\def\PY@tc##1{\textcolor[rgb]{0.00,0.00,0.50}{##1}}}
\@namedef{PY@tok@gu}{\let\PY@bf=\textbf\def\PY@tc##1{\textcolor[rgb]{0.50,0.00,0.50}{##1}}}
\@namedef{PY@tok@gd}{\def\PY@tc##1{\textcolor[rgb]{0.63,0.00,0.00}{##1}}}
\@namedef{PY@tok@gi}{\def\PY@tc##1{\textcolor[rgb]{0.00,0.52,0.00}{##1}}}
\@namedef{PY@tok@gr}{\def\PY@tc##1{\textcolor[rgb]{0.89,0.00,0.00}{##1}}}
\@namedef{PY@tok@ge}{\let\PY@it=\textit}
\@namedef{PY@tok@gs}{\let\PY@bf=\textbf}
\@namedef{PY@tok@ges}{\let\PY@bf=\textbf\let\PY@it=\textit}
\@namedef{PY@tok@gp}{\let\PY@bf=\textbf\def\PY@tc##1{\textcolor[rgb]{0.00,0.00,0.50}{##1}}}
\@namedef{PY@tok@go}{\def\PY@tc##1{\textcolor[rgb]{0.44,0.44,0.44}{##1}}}
\@namedef{PY@tok@gt}{\def\PY@tc##1{\textcolor[rgb]{0.00,0.27,0.87}{##1}}}
\@namedef{PY@tok@err}{\def\PY@bc##1{{\setlength{\fboxsep}{\string -\fboxrule}\fcolorbox[rgb]{1.00,0.00,0.00}{1,1,1}{\strut ##1}}}}
\@namedef{PY@tok@kc}{\let\PY@bf=\textbf\def\PY@tc##1{\textcolor[rgb]{0.00,0.50,0.00}{##1}}}
\@namedef{PY@tok@kd}{\let\PY@bf=\textbf\def\PY@tc##1{\textcolor[rgb]{0.00,0.50,0.00}{##1}}}
\@namedef{PY@tok@kn}{\let\PY@bf=\textbf\def\PY@tc##1{\textcolor[rgb]{0.00,0.50,0.00}{##1}}}
\@namedef{PY@tok@kr}{\let\PY@bf=\textbf\def\PY@tc##1{\textcolor[rgb]{0.00,0.50,0.00}{##1}}}
\@namedef{PY@tok@bp}{\def\PY@tc##1{\textcolor[rgb]{0.00,0.50,0.00}{##1}}}
\@namedef{PY@tok@fm}{\def\PY@tc##1{\textcolor[rgb]{0.00,0.00,1.00}{##1}}}
\@namedef{PY@tok@vc}{\def\PY@tc##1{\textcolor[rgb]{0.10,0.09,0.49}{##1}}}
\@namedef{PY@tok@vg}{\def\PY@tc##1{\textcolor[rgb]{0.10,0.09,0.49}{##1}}}
\@namedef{PY@tok@vi}{\def\PY@tc##1{\textcolor[rgb]{0.10,0.09,0.49}{##1}}}
\@namedef{PY@tok@vm}{\def\PY@tc##1{\textcolor[rgb]{0.10,0.09,0.49}{##1}}}
\@namedef{PY@tok@sa}{\def\PY@tc##1{\textcolor[rgb]{0.73,0.13,0.13}{##1}}}
\@namedef{PY@tok@sb}{\def\PY@tc##1{\textcolor[rgb]{0.73,0.13,0.13}{##1}}}
\@namedef{PY@tok@sc}{\def\PY@tc##1{\textcolor[rgb]{0.73,0.13,0.13}{##1}}}
\@namedef{PY@tok@dl}{\def\PY@tc##1{\textcolor[rgb]{0.73,0.13,0.13}{##1}}}
\@namedef{PY@tok@s2}{\def\PY@tc##1{\textcolor[rgb]{0.73,0.13,0.13}{##1}}}
\@namedef{PY@tok@sh}{\def\PY@tc##1{\textcolor[rgb]{0.73,0.13,0.13}{##1}}}
\@namedef{PY@tok@s1}{\def\PY@tc##1{\textcolor[rgb]{0.73,0.13,0.13}{##1}}}
\@namedef{PY@tok@mb}{\def\PY@tc##1{\textcolor[rgb]{0.40,0.40,0.40}{##1}}}
\@namedef{PY@tok@mf}{\def\PY@tc##1{\textcolor[rgb]{0.40,0.40,0.40}{##1}}}
\@namedef{PY@tok@mh}{\def\PY@tc##1{\textcolor[rgb]{0.40,0.40,0.40}{##1}}}
\@namedef{PY@tok@mi}{\def\PY@tc##1{\textcolor[rgb]{0.40,0.40,0.40}{##1}}}
\@namedef{PY@tok@il}{\def\PY@tc##1{\textcolor[rgb]{0.40,0.40,0.40}{##1}}}
\@namedef{PY@tok@mo}{\def\PY@tc##1{\textcolor[rgb]{0.40,0.40,0.40}{##1}}}
\@namedef{PY@tok@ch}{\let\PY@it=\textit\def\PY@tc##1{\textcolor[rgb]{0.24,0.48,0.48}{##1}}}
\@namedef{PY@tok@cm}{\let\PY@it=\textit\def\PY@tc##1{\textcolor[rgb]{0.24,0.48,0.48}{##1}}}
\@namedef{PY@tok@cpf}{\let\PY@it=\textit\def\PY@tc##1{\textcolor[rgb]{0.24,0.48,0.48}{##1}}}
\@namedef{PY@tok@c1}{\let\PY@it=\textit\def\PY@tc##1{\textcolor[rgb]{0.24,0.48,0.48}{##1}}}
\@namedef{PY@tok@cs}{\let\PY@it=\textit\def\PY@tc##1{\textcolor[rgb]{0.24,0.48,0.48}{##1}}}

\def\PYZbs{\char`\\}
\def\PYZus{\char`\_}
\def\PYZob{\char`\{}
\def\PYZcb{\char`\}}
\def\PYZca{\char`\^}
\def\PYZam{\char`\&}
\def\PYZlt{\char`\<}
\def\PYZgt{\char`\>}
\def\PYZsh{\char`\#}
\def\PYZpc{\char`\%}
\def\PYZdl{\char`\$}
\def\PYZhy{\char`\-}
\def\PYZsq{\char`\'}
\def\PYZdq{\char`\"}
\def\PYZti{\char`\~}
% for compatibility with earlier versions
\def\PYZat{@}
\def\PYZlb{[}
\def\PYZrb{]}
\makeatother


    % For linebreaks inside Verbatim environment from package fancyvrb.
    \makeatletter
        \newbox\Wrappedcontinuationbox
        \newbox\Wrappedvisiblespacebox
        \newcommand*\Wrappedvisiblespace {\textcolor{red}{\textvisiblespace}}
        \newcommand*\Wrappedcontinuationsymbol {\textcolor{red}{\llap{\tiny$\m@th\hookrightarrow$}}}
        \newcommand*\Wrappedcontinuationindent {3ex }
        \newcommand*\Wrappedafterbreak {\kern\Wrappedcontinuationindent\copy\Wrappedcontinuationbox}
        % Take advantage of the already applied Pygments mark-up to insert
        % potential linebreaks for TeX processing.
        %        {, <, #, %, $, ' and ": go to next line.
        %        _, }, ^, &, >, - and ~: stay at end of broken line.
        % Use of \textquotesingle for straight quote.
        \newcommand*\Wrappedbreaksatspecials {%
            \def\PYGZus{\discretionary{\char`\_}{\Wrappedafterbreak}{\char`\_}}%
            \def\PYGZob{\discretionary{}{\Wrappedafterbreak\char`\{}{\char`\{}}%
            \def\PYGZcb{\discretionary{\char`\}}{\Wrappedafterbreak}{\char`\}}}%
            \def\PYGZca{\discretionary{\char`\^}{\Wrappedafterbreak}{\char`\^}}%
            \def\PYGZam{\discretionary{\char`\&}{\Wrappedafterbreak}{\char`\&}}%
            \def\PYGZlt{\discretionary{}{\Wrappedafterbreak\char`\<}{\char`\<}}%
            \def\PYGZgt{\discretionary{\char`\>}{\Wrappedafterbreak}{\char`\>}}%
            \def\PYGZsh{\discretionary{}{\Wrappedafterbreak\char`\#}{\char`\#}}%
            \def\PYGZpc{\discretionary{}{\Wrappedafterbreak\char`\%}{\char`\%}}%
            \def\PYGZdl{\discretionary{}{\Wrappedafterbreak\char`\$}{\char`\$}}%
            \def\PYGZhy{\discretionary{\char`\-}{\Wrappedafterbreak}{\char`\-}}%
            \def\PYGZsq{\discretionary{}{\Wrappedafterbreak\textquotesingle}{\textquotesingle}}%
            \def\PYGZdq{\discretionary{}{\Wrappedafterbreak\char`\"}{\char`\"}}%
            \def\PYGZti{\discretionary{\char`\~}{\Wrappedafterbreak}{\char`\~}}%
        }
        % Some characters . , ; ? ! / are not pygmentized.
        % This macro makes them "active" and they will insert potential linebreaks
        \newcommand*\Wrappedbreaksatpunct {%
            \lccode`\~`\.\lowercase{\def~}{\discretionary{\hbox{\char`\.}}{\Wrappedafterbreak}{\hbox{\char`\.}}}%
            \lccode`\~`\,\lowercase{\def~}{\discretionary{\hbox{\char`\,}}{\Wrappedafterbreak}{\hbox{\char`\,}}}%
            \lccode`\~`\;\lowercase{\def~}{\discretionary{\hbox{\char`\;}}{\Wrappedafterbreak}{\hbox{\char`\;}}}%
            \lccode`\~`\:\lowercase{\def~}{\discretionary{\hbox{\char`\:}}{\Wrappedafterbreak}{\hbox{\char`\:}}}%
            \lccode`\~`\?\lowercase{\def~}{\discretionary{\hbox{\char`\?}}{\Wrappedafterbreak}{\hbox{\char`\?}}}%
            \lccode`\~`\!\lowercase{\def~}{\discretionary{\hbox{\char`\!}}{\Wrappedafterbreak}{\hbox{\char`\!}}}%
            \lccode`\~`\/\lowercase{\def~}{\discretionary{\hbox{\char`\/}}{\Wrappedafterbreak}{\hbox{\char`\/}}}%
            \catcode`\.\active
            \catcode`\,\active
            \catcode`\;\active
            \catcode`\:\active
            \catcode`\?\active
            \catcode`\!\active
            \catcode`\/\active
            \lccode`\~`\~
        }
    \makeatother

    \let\OriginalVerbatim=\Verbatim
    \makeatletter
    \renewcommand{\Verbatim}[1][1]{%
        %\parskip\z@skip
        \sbox\Wrappedcontinuationbox {\Wrappedcontinuationsymbol}%
        \sbox\Wrappedvisiblespacebox {\FV@SetupFont\Wrappedvisiblespace}%
        \def\FancyVerbFormatLine ##1{\hsize\linewidth
            \vtop{\raggedright\hyphenpenalty\z@\exhyphenpenalty\z@
                \doublehyphendemerits\z@\finalhyphendemerits\z@
                \strut ##1\strut}%
        }%
        % If the linebreak is at a space, the latter will be displayed as visible
        % space at end of first line, and a continuation symbol starts next line.
        % Stretch/shrink are however usually zero for typewriter font.
        \def\FV@Space {%
            \nobreak\hskip\z@ plus\fontdimen3\font minus\fontdimen4\font
            \discretionary{\copy\Wrappedvisiblespacebox}{\Wrappedafterbreak}
            {\kern\fontdimen2\font}%
        }%

        % Allow breaks at special characters using \PYG... macros.
        \Wrappedbreaksatspecials
        % Breaks at punctuation characters . , ; ? ! and / need catcode=\active
        \OriginalVerbatim[#1,codes*=\Wrappedbreaksatpunct]%
    }
    \makeatother

    % Exact colors from NB
    \definecolor{incolor}{HTML}{303F9F}
    \definecolor{outcolor}{HTML}{D84315}
    \definecolor{cellborder}{HTML}{CFCFCF}
    \definecolor{cellbackground}{HTML}{F7F7F7}

    % prompt
    \makeatletter
    \newcommand{\boxspacing}{\kern\kvtcb@left@rule\kern\kvtcb@boxsep}
    \makeatother
    \newcommand{\prompt}[4]{
        {\ttfamily\llap{{\color{#2}[#3]:\hspace{3pt}#4}}\vspace{-\baselineskip}}
    }
    

    
    % Prevent overflowing lines due to hard-to-break entities
    \sloppy
    % Setup hyperref package
    \hypersetup{
      breaklinks=true,  % so long urls are correctly broken across lines
      colorlinks=true,
      urlcolor=urlcolor,
      linkcolor=linkcolor,
      citecolor=citecolor,
      }
    % Slightly bigger margins than the latex defaults
    
    \geometry{verbose,tmargin=1in,bmargin=1in,lmargin=1in,rmargin=1in}
    
    

\begin{document}
    
    \maketitle
    
    

    
    \section{Titanic - analiza danych o
pasażerach}\label{titanic---analiza-danych-o-pasaux17cerach}

    \subsection{O Danych}\label{o-danych}

Dane o pasażerach Titanica

Zbiór danych zawiera informacje o pasażerach RMS Titanic, który zatonął
15 kwietnia 1912 roku po zderzeniu z górą lodową. Dane obejmują takie
atrybuty jak klasa podróży, wiek, płeć, liczba rodzeństwa/małżonków na
pokładzie, liczba rodziców/dzieci na pokładzie, cena biletu oraz miejsce
zaokrętowania.

Zbiór zawiera także informację o tym, czy pasażer przeżył katastrofę.

Titanic przewoził ponad 2,200 osób, z czego ponad 1,500 zginęło, co
czyni tę katastrofę jedną z najbardziej tragicznych w historii morskiej.

    \subsection{O Danych}\label{o-danych}

Kolumny: * \textbf{pclass} - Klasa biletu * \textbf{survived} - Czy
pasażer przeżył katastrofę * \textbf{name} - Imię i nazwisko pasażera *
\textbf{sex} - Płeć pasażera * \textbf{age} - Wiek pasażera *
\textbf{sibsp} - Liczba rodzeństwa/małżonków na pokładzie *
\textbf{parch} - Liczba rodziców/dzieci na pokładzie * \textbf{ticket} -
Numer biletu * \textbf{fare} - Cena biletu * \textbf{cabin} - Numer
kabiny * \textbf{embarked} - Port, w którym pasażer wszedł na pokład (C
= Cherbourg, Q = Queenstown, S = Southampton) * \textbf{boat} - Numer
łodzi ratunkowej * \textbf{body} - Numer ciała (jeśli pasażer nie
przeżył i ciało zostało odnalezione) * \textbf{home.dest} - Miejsce
docelowe

    \begin{tcolorbox}[breakable, size=fbox, boxrule=1pt, pad at break*=1mm,colback=cellbackground, colframe=cellborder]
\prompt{In}{incolor}{63}{\boxspacing}
\begin{Verbatim}[commandchars=\\\{\}]
\PY{k+kn}{import}\PY{+w}{ }\PY{n+nn}{pandas}\PY{+w}{ }\PY{k}{as}\PY{+w}{ }\PY{n+nn}{pd}
\PY{k+kn}{import}\PY{+w}{ }\PY{n+nn}{matplotlib}\PY{n+nn}{.}\PY{n+nn}{pyplot}\PY{+w}{ }\PY{k}{as}\PY{+w}{ }\PY{n+nn}{plt}
\PY{k+kn}{import}\PY{+w}{ }\PY{n+nn}{seaborn}\PY{+w}{ }\PY{k}{as}\PY{+w}{ }\PY{n+nn}{sns}
\PY{k+kn}{from}\PY{+w}{ }\PY{n+nn}{itables}\PY{+w}{ }\PY{k+kn}{import} \PY{n}{show}
\PY{k+kn}{from}\PY{+w}{ }\PY{n+nn}{statsmodels}\PY{n+nn}{.}\PY{n+nn}{graphics}\PY{n+nn}{.}\PY{n+nn}{mosaicplot}\PY{+w}{ }\PY{k+kn}{import} \PY{n}{mosaic}
\end{Verbatim}
\end{tcolorbox}

    \section{1. Przegląd i analiza danych dotyczących Titanica i jego
pasażerów.}\label{przeglux105d-i-analiza-danych-dotyczux105cych-titanica-i-jego-pasaux17ceruxf3w.}

    \subsection{1.1 Wczytanie danych i przegląd losowych
wartości.}\label{wczytanie-danych-i-przeglux105d-losowych-wartoux15bci.}

    \begin{tcolorbox}[breakable, size=fbox, boxrule=1pt, pad at break*=1mm,colback=cellbackground, colframe=cellborder]
\prompt{In}{incolor}{64}{\boxspacing}
\begin{Verbatim}[commandchars=\\\{\}]
\PY{n}{df} \PY{o}{=} \PY{n}{pd}\PY{o}{.}\PY{n}{read\PYZus{}csv}\PY{p}{(}\PY{l+s+s1}{\PYZsq{}}\PY{l+s+s1}{26\PYZus{}\PYZus{}titanic.csv}\PY{l+s+s1}{\PYZsq{}}\PY{p}{,} \PY{n}{sep}\PY{o}{=}\PY{l+s+s2}{\PYZdq{}}\PY{l+s+s2}{,}\PY{l+s+s2}{\PYZdq{}}\PY{p}{)}
\PY{n}{df}
\end{Verbatim}
\end{tcolorbox}

            \begin{tcolorbox}[breakable, size=fbox, boxrule=.5pt, pad at break*=1mm, opacityfill=0]
\prompt{Out}{outcolor}{64}{\boxspacing}
\begin{Verbatim}[commandchars=\\\{\}]
      pclass  survived                                             name  \textbackslash{}
0        1.0       1.0                    Allen, Miss. Elisabeth Walton
1        1.0       1.0                   Allison, Master. Hudson Trevor
2        1.0       0.0                     Allison, Miss. Helen Loraine
3        1.0       0.0             Allison, Mr. Hudson Joshua Creighton
4        1.0       0.0  Allison, Mrs. Hudson J C (Bessie Waldo Daniels)
{\ldots}      {\ldots}       {\ldots}                                              {\ldots}
1305     3.0       0.0                            Zabour, Miss. Thamine
1306     3.0       0.0                        Zakarian, Mr. Mapriededer
1307     3.0       0.0                              Zakarian, Mr. Ortin
1308     3.0       0.0                               Zimmerman, Mr. Leo
1309     NaN       NaN                                              NaN

         sex      age  sibsp  parch  ticket      fare    cabin embarked boat  \textbackslash{}
0     female  29.0000    0.0    0.0   24160  211.3375       B5        S    2
1       male   0.9167    1.0    2.0  113781  151.5500  C22 C26        S   11
2     female   2.0000    1.0    2.0  113781  151.5500  C22 C26        S  NaN
3       male  30.0000    1.0    2.0  113781  151.5500  C22 C26        S  NaN
4     female  25.0000    1.0    2.0  113781  151.5500  C22 C26        S  NaN
{\ldots}      {\ldots}      {\ldots}    {\ldots}    {\ldots}     {\ldots}       {\ldots}      {\ldots}      {\ldots}  {\ldots}
1305  female      NaN    1.0    0.0    2665   14.4542      NaN        C  NaN
1306    male  26.5000    0.0    0.0    2656    7.2250      NaN        C  NaN
1307    male  27.0000    0.0    0.0    2670    7.2250      NaN        C  NaN
1308    male  29.0000    0.0    0.0  315082    7.8750      NaN        S  NaN
1309     NaN      NaN    NaN    NaN     NaN       NaN      NaN      NaN  NaN

       body                        home.dest
0       NaN                     St Louis, MO
1       NaN  Montreal, PQ / Chesterville, ON
2       NaN  Montreal, PQ / Chesterville, ON
3     135.0  Montreal, PQ / Chesterville, ON
4       NaN  Montreal, PQ / Chesterville, ON
{\ldots}     {\ldots}                              {\ldots}
1305    NaN                              NaN
1306  304.0                              NaN
1307    NaN                              NaN
1308    NaN                              NaN
1309    NaN                              NaN

[1310 rows x 14 columns]
\end{Verbatim}
\end{tcolorbox}
        
    \begin{tcolorbox}[breakable, size=fbox, boxrule=1pt, pad at break*=1mm,colback=cellbackground, colframe=cellborder]
\prompt{In}{incolor}{65}{\boxspacing}
\begin{Verbatim}[commandchars=\\\{\}]
\PY{n}{df}\PY{o}{.}\PY{n}{info}\PY{p}{(}\PY{p}{)}
\end{Verbatim}
\end{tcolorbox}

    \begin{Verbatim}[commandchars=\\\{\}]
<class 'pandas.core.frame.DataFrame'>
RangeIndex: 1310 entries, 0 to 1309
Data columns (total 14 columns):
 \#   Column     Non-Null Count  Dtype
---  ------     --------------  -----
 0   pclass     1309 non-null   float64
 1   survived   1309 non-null   float64
 2   name       1309 non-null   object
 3   sex        1309 non-null   object
 4   age        1046 non-null   float64
 5   sibsp      1309 non-null   float64
 6   parch      1309 non-null   float64
 7   ticket     1309 non-null   object
 8   fare       1308 non-null   float64
 9   cabin      295 non-null    object
 10  embarked   1307 non-null   object
 11  boat       486 non-null    object
 12  body       121 non-null    float64
 13  home.dest  745 non-null    object
dtypes: float64(7), object(7)
memory usage: 143.4+ KB
    \end{Verbatim}

    \begin{tcolorbox}[breakable, size=fbox, boxrule=1pt, pad at break*=1mm,colback=cellbackground, colframe=cellborder]
\prompt{In}{incolor}{66}{\boxspacing}
\begin{Verbatim}[commandchars=\\\{\}]
\PY{n}{df}\PY{p}{[}\PY{p}{[}\PY{l+s+s1}{\PYZsq{}}\PY{l+s+s1}{pclass}\PY{l+s+s1}{\PYZsq{}}\PY{p}{,} \PY{l+s+s1}{\PYZsq{}}\PY{l+s+s1}{survived}\PY{l+s+s1}{\PYZsq{}}\PY{p}{,} \PY{l+s+s1}{\PYZsq{}}\PY{l+s+s1}{name}\PY{l+s+s1}{\PYZsq{}}\PY{p}{,} \PY{l+s+s1}{\PYZsq{}}\PY{l+s+s1}{sex}\PY{l+s+s1}{\PYZsq{}}\PY{p}{,} \PY{l+s+s1}{\PYZsq{}}\PY{l+s+s1}{age}\PY{l+s+s1}{\PYZsq{}}\PY{p}{,} \PY{l+s+s1}{\PYZsq{}}\PY{l+s+s1}{sibsp}\PY{l+s+s1}{\PYZsq{}}\PY{p}{,} \PY{l+s+s1}{\PYZsq{}}\PY{l+s+s1}{parch}\PY{l+s+s1}{\PYZsq{}}\PY{p}{]}\PY{p}{]}\PY{o}{.}\PY{n}{sample}\PY{p}{(}\PY{l+m+mi}{6}\PY{p}{,} \PY{n}{random\PYZus{}state}\PY{o}{=}\PY{l+m+mi}{42}\PY{p}{)}
\end{Verbatim}
\end{tcolorbox}

            \begin{tcolorbox}[breakable, size=fbox, boxrule=.5pt, pad at break*=1mm, opacityfill=0]
\prompt{Out}{outcolor}{66}{\boxspacing}
\begin{Verbatim}[commandchars=\\\{\}]
     pclass  survived                           name     sex   age  sibsp  \textbackslash{}
701     3.0       0.0               Calic, Mr. Petar    male  17.0    0.0
994     3.0       0.0        Mardirosian, Mr. Sarkis    male   NaN    0.0
350     2.0       1.0      Brown, Miss. Edith Eileen  female  15.0    0.0
986     3.0       0.0  Maenpaa, Mr. Matti Alexanteri    male  22.0    0.0
409     2.0       0.0        Fox, Mr. Stanley Hubert    male  36.0    0.0
917     3.0       1.0               Karun, Mr. Franz    male  39.0    0.0

     parch
701    0.0
994    0.0
350    2.0
986    0.0
409    0.0
917    1.0
\end{Verbatim}
\end{tcolorbox}
        
    \begin{tcolorbox}[breakable, size=fbox, boxrule=1pt, pad at break*=1mm,colback=cellbackground, colframe=cellborder]
\prompt{In}{incolor}{67}{\boxspacing}
\begin{Verbatim}[commandchars=\\\{\}]
\PY{n}{df}\PY{p}{[}\PY{p}{[}\PY{l+s+s1}{\PYZsq{}}\PY{l+s+s1}{ticket}\PY{l+s+s1}{\PYZsq{}}\PY{p}{,} \PY{l+s+s1}{\PYZsq{}}\PY{l+s+s1}{fare}\PY{l+s+s1}{\PYZsq{}}\PY{p}{,} \PY{l+s+s1}{\PYZsq{}}\PY{l+s+s1}{cabin}\PY{l+s+s1}{\PYZsq{}}\PY{p}{,} \PY{l+s+s1}{\PYZsq{}}\PY{l+s+s1}{embarked}\PY{l+s+s1}{\PYZsq{}}\PY{p}{,} \PY{l+s+s1}{\PYZsq{}}\PY{l+s+s1}{boat}\PY{l+s+s1}{\PYZsq{}}\PY{p}{,} \PY{l+s+s1}{\PYZsq{}}\PY{l+s+s1}{body}\PY{l+s+s1}{\PYZsq{}}\PY{p}{,} \PY{l+s+s1}{\PYZsq{}}\PY{l+s+s1}{home.dest}\PY{l+s+s1}{\PYZsq{}}\PY{p}{]}\PY{p}{]}\PY{o}{.}\PY{n}{sample}\PY{p}{(}\PY{l+m+mi}{6}\PY{p}{,} \PY{n}{random\PYZus{}state}\PY{o}{=}\PY{l+m+mi}{42}\PY{p}{)}
\end{Verbatim}
\end{tcolorbox}

            \begin{tcolorbox}[breakable, size=fbox, boxrule=.5pt, pad at break*=1mm, opacityfill=0]
\prompt{Out}{outcolor}{67}{\boxspacing}
\begin{Verbatim}[commandchars=\\\{\}]
                ticket     fare  cabin embarked boat   body  \textbackslash{}
701             315086   8.6625    NaN        S  NaN    NaN
994               2655   7.2292  F E46        C  NaN    NaN
350              29750  39.0000    NaN        S   14    NaN
986  STON/O 2. 3101275   7.1250    NaN        S  NaN    NaN
409             229236  13.0000    NaN        S  NaN  236.0
917             349256  13.4167    NaN        C   15    NaN

                                 home.dest
701                                    NaN
994                                    NaN
350  Cape Town, South Africa / Seattle, WA
986                                    NaN
409                          Rochester, NY
917                                    NaN
\end{Verbatim}
\end{tcolorbox}
        
    \paragraph{Po wczytaniu danych mamy informację o 1310 wierszach i 14
kolumnach.}\label{po-wczytaniu-danych-mamy-informacjux119-o-1310-wierszach-i-14-kolumnach.}

\paragraph{Zauważyć można, że w wierszu 1309, we wszystkich kolumnach są
puste wartości, należy zatem usunąć ten wiersz przed przystąpieniem, do
dalszej
analizy.}\label{zauwaux17cyux107-moux17cna-ux17ce-w-wierszu-1309-we-wszystkich-kolumnach-sux105-puste-wartoux15bci-naleux17cy-zatem-usunux105ux107-ten-wiersz-przed-przystux105pieniem-do-dalszej-analizy.}

\paragraph{Po przeglądzie losowych wartości widać, że istnieje wiele
pustych wartości w niektórych kolumnach. W dalszej analizie, należy
zastanowić się, czy brakujące wartości będą miały istotny wpływ na
wyniki analizy i czy będzie potrzeba wypełnienia tych
wartości.}\label{po-przeglux105dzie-losowych-wartoux15bci-widaux107-ux17ce-istnieje-wiele-pustych-wartoux15bci-w-niektuxf3rych-kolumnach.-w-dalszej-analizie-naleux17cy-zastanowiux107-siux119-czy-brakujux105ce-wartoux15bci-bux119dux105-miaux142y-istotny-wpux142yw-na-wyniki-analizy-i-czy-bux119dzie-potrzeba-wypeux142nienia-tych-wartoux15bci.}

    \paragraph{Jeden z wierszy ma puste wartości we wszystkich
kolumnach}\label{jeden-z-wierszy-ma-puste-wartoux15bci-we-wszystkich-kolumnach}

    \begin{tcolorbox}[breakable, size=fbox, boxrule=1pt, pad at break*=1mm,colback=cellbackground, colframe=cellborder]
\prompt{In}{incolor}{68}{\boxspacing}
\begin{Verbatim}[commandchars=\\\{\}]
\PY{n}{empty\PYZus{}rows} \PY{o}{=} \PY{n}{df}\PY{p}{[}\PY{n}{df}\PY{o}{.}\PY{n}{isnull}\PY{p}{(}\PY{p}{)}\PY{o}{.}\PY{n}{all}\PY{p}{(}\PY{n}{axis}\PY{o}{=}\PY{l+m+mi}{1}\PY{p}{)}\PY{p}{]}
\PY{n}{empty\PYZus{}rows}
\end{Verbatim}
\end{tcolorbox}

            \begin{tcolorbox}[breakable, size=fbox, boxrule=.5pt, pad at break*=1mm, opacityfill=0]
\prompt{Out}{outcolor}{68}{\boxspacing}
\begin{Verbatim}[commandchars=\\\{\}]
      pclass  survived name  sex  age  sibsp  parch ticket  fare cabin  \textbackslash{}
1309     NaN       NaN  NaN  NaN  NaN    NaN    NaN    NaN   NaN   NaN

     embarked boat  body home.dest
1309      NaN  NaN   NaN       NaN
\end{Verbatim}
\end{tcolorbox}
        
    \paragraph{Usuwam wiersz z pustymi
wartościami.}\label{usuwam-wiersz-z-pustymi-wartoux15bciami.}

    \begin{tcolorbox}[breakable, size=fbox, boxrule=1pt, pad at break*=1mm,colback=cellbackground, colframe=cellborder]
\prompt{In}{incolor}{69}{\boxspacing}
\begin{Verbatim}[commandchars=\\\{\}]
\PY{n}{df} \PY{o}{=} \PY{n}{df}\PY{o}{.}\PY{n}{dropna}\PY{p}{(}\PY{n}{how} \PY{o}{=} \PY{l+s+s1}{\PYZsq{}}\PY{l+s+s1}{all}\PY{l+s+s1}{\PYZsq{}}\PY{p}{)}
\end{Verbatim}
\end{tcolorbox}

    \subsection{1.2 Sprawdzenie wartości
unikatowych.}\label{sprawdzenie-wartoux15bci-unikatowych.}

    \begin{tcolorbox}[breakable, size=fbox, boxrule=1pt, pad at break*=1mm,colback=cellbackground, colframe=cellborder]
\prompt{In}{incolor}{70}{\boxspacing}
\begin{Verbatim}[commandchars=\\\{\}]
\PY{n}{pd}\PY{o}{.}\PY{n}{DataFrame}\PY{p}{(}\PY{n}{df}\PY{o}{.}\PY{n}{nunique}\PY{p}{(}\PY{p}{)}\PY{p}{)}
\end{Verbatim}
\end{tcolorbox}

            \begin{tcolorbox}[breakable, size=fbox, boxrule=.5pt, pad at break*=1mm, opacityfill=0]
\prompt{Out}{outcolor}{70}{\boxspacing}
\begin{Verbatim}[commandchars=\\\{\}]
              0
pclass        3
survived      2
name       1307
sex           2
age          98
sibsp         7
parch         8
ticket      929
fare        281
cabin       186
embarked      3
boat         27
body        121
home.dest   369
\end{Verbatim}
\end{tcolorbox}
        
    \paragraph{Krótkie spostrzeżenia o wartościach
unikatowych:}\label{kruxf3tkie-spostrzeux17cenia-o-wartoux15bciach-unikatowych}

\begin{itemize}
\tightlist
\item
  \textbf{pclass} - 3 klasy biletów (ilu pasażerów w każdej klasie)
\item
  \textbf{survived} - 2 wartości oznaczające czy pasażer ocalał, czy
  nie(sprawdzić ilu ocalonych)
\item
  \textbf{name} - 1307 nazwisk na 1309 rekordów (sprawdzić duplikaty)
\item
  \textbf{sex} - 2 wartości oznaczające płeć (sprawdzić ile
  kobiet/mężczyzn)
\item
  \textbf{age} - 98 wartości określających wiek (w losowych danych widać
  wiek podany jako ułamek, zamienić na liczby całkowite, ponownie
  sprawdzić wartości unikatowe)
\item
  \textbf{sibsp} - 7 wartości dla liczby rodzeństwa/małżonków na
  pokładzie
\item
  \textbf{parch} - 8 wartości dla rodziców/dzieci na pokładzie
\item
  \textbf{ticket} - 929 wartości z numerem biletu (sprawdzić duplikaty,
  dlaczego występują)
\item
  \textbf{fare} - 281 wartości z różną ceną biletu(sprawdzić od czego
  uzależniona cena)
\item
  \textbf{cabin} - 186 numerów kabin
\item
  \textbf{embarked} - 3 rożne porty wejścia pasażerów na pokład
\item
  \textbf{boat} - 27 numerów łodzi ratunkowych(jakieś zależności?)
\item
  \textbf{body} - 121 wartości dla odnalezionych ciał ofiar katastrofy
\item
  \textbf{home.dest} - 369 wartości dla celu podróży pasażerów(sprawdzić
  korelację ocalony cel podrózy)
\end{itemize}

    \subsection{1.3 Przegląd danych
statystycznych.}\label{przeglux105d-danych-statystycznych.}

    \begin{tcolorbox}[breakable, size=fbox, boxrule=1pt, pad at break*=1mm,colback=cellbackground, colframe=cellborder]
\prompt{In}{incolor}{71}{\boxspacing}
\begin{Verbatim}[commandchars=\\\{\}]
\PY{n}{df}\PY{o}{.}\PY{n}{describe}\PY{p}{(}\PY{p}{)}
\end{Verbatim}
\end{tcolorbox}

            \begin{tcolorbox}[breakable, size=fbox, boxrule=.5pt, pad at break*=1mm, opacityfill=0]
\prompt{Out}{outcolor}{71}{\boxspacing}
\begin{Verbatim}[commandchars=\\\{\}]
            pclass     survived          age        sibsp        parch  \textbackslash{}
count  1309.000000  1309.000000  1046.000000  1309.000000  1309.000000
mean      2.294882     0.381971    29.881135     0.498854     0.385027
std       0.837836     0.486055    14.413500     1.041658     0.865560
min       1.000000     0.000000     0.166700     0.000000     0.000000
25\%       2.000000     0.000000    21.000000     0.000000     0.000000
50\%       3.000000     0.000000    28.000000     0.000000     0.000000
75\%       3.000000     1.000000    39.000000     1.000000     0.000000
max       3.000000     1.000000    80.000000     8.000000     9.000000

              fare        body
count  1308.000000  121.000000
mean     33.295479  160.809917
std      51.758668   97.696922
min       0.000000    1.000000
25\%       7.895800   72.000000
50\%      14.454200  155.000000
75\%      31.275000  256.000000
max     512.329200  328.000000
\end{Verbatim}
\end{tcolorbox}
        
    \subparagraph{Mamy 7 kolumn numerycznych, przechowujących dane o klasie
bilety, ocalałych, wieku, rodzeństwa/małżonków, rodziców/dzeici, cenie
biletu, odnalezionym ciele
ofiary.}\label{mamy-7-kolumn-numerycznych-przechowujux105cych-dane-o-klasie-bilety-ocalaux142ych-wieku-rodzeux144stwamaux142ux17conkuxf3w-rodzicuxf3wdzeici-cenie-biletu-odnalezionym-ciele-ofiary.}

\subparagraph{Katastrofę przeżyło 38\%
pasażerów.}\label{katastrofux119-przeux17cyux142o-38-pasaux17ceruxf3w.}

\subparagraph{Najmłodszy z pasażerów miał mniej niż rok, najstarszy 80
lat, średni wiek to ok 30
lat.}\label{najmux142odszy-z-pasaux17ceruxf3w-miaux142-mniej-niux17c-rok-najstarszy-80-lat-ux15bredni-wiek-to-ok-30-lat.}

\subparagraph{49\% pasażerów podróżowało z małżonkiem lub
rodzeństwem.}\label{pasaux17ceruxf3w-podruxf3ux17cowaux142o-z-maux142ux17conkiem-lub-rodzeux144stwem.}

\subparagraph{38\% pasażerów było
rodzicami/dziećmi}\label{pasaux17ceruxf3w-byux142o-rodzicamidzieux107mi}

\subparagraph{Średnia cena biletu to 33. najtańszy bilet kosztował 0,
najdroższy
512.}\label{ux15brednia-cena-biletu-to-33.-najtaux144szy-bilet-kosztowaux142-0-najdroux17cszy-512.}

\subparagraph{Odnaleziono 121 ciał.}\label{odnaleziono-121-ciaux142.}

    \section{2 Analiza brakujących
wartości.}\label{analiza-brakujux105cych-wartoux15bci.}

    \begin{tcolorbox}[breakable, size=fbox, boxrule=1pt, pad at break*=1mm,colback=cellbackground, colframe=cellborder]
\prompt{In}{incolor}{72}{\boxspacing}
\begin{Verbatim}[commandchars=\\\{\}]
\PY{n}{pd}\PY{o}{.}\PY{n}{DataFrame}\PY{p}{(}\PY{n}{df}\PY{o}{.}\PY{n}{isnull}\PY{p}{(}\PY{p}{)}\PY{o}{.}\PY{n}{sum}\PY{p}{(}\PY{p}{)}\PY{p}{)}
\end{Verbatim}
\end{tcolorbox}

            \begin{tcolorbox}[breakable, size=fbox, boxrule=.5pt, pad at break*=1mm, opacityfill=0]
\prompt{Out}{outcolor}{72}{\boxspacing}
\begin{Verbatim}[commandchars=\\\{\}]
              0
pclass        0
survived      0
name          0
sex           0
age         263
sibsp         0
parch         0
ticket        0
fare          1
cabin      1014
embarked      2
boat        823
body       1188
home.dest   564
\end{Verbatim}
\end{tcolorbox}
        
    \paragraph{Brakujące dane:}\label{brakujux105ce-dane}

\begin{itemize}
\tightlist
\item
  \textbf{age} 263 dane o wieku (naprawić średnią dla mężczyzn i
  kobiet?)
\item
  \textbf{fare} 1 cena biletu (naprawić średnią ceną)
\item
  \textbf{cabin} 1014 danych o numerze kabiny
\item
  \textbf{embarked} 2 informacje o porcie wejścia pasażerów na pokład
\item
  \textbf{boat} 823 numer łodzi ratunkowej, w której przebywał pasażer
  (sprawdzic brakujące wartośći dla ocalałych pasażerów)
\item
  \textbf{body} 1188 numer ciała
\item
  \textbf{home.dst} 564 celu podróży.
\end{itemize}

    \section{3 Analiza poszczególnych
danych.}\label{analiza-poszczeguxf3lnych-danych.}

    \paragraph{PCLASS - ilość pasażerów w każdej
klasie}\label{pclass---iloux15bux107-pasaux17ceruxf3w-w-kaux17cdej-klasie}

    \begin{tcolorbox}[breakable, size=fbox, boxrule=1pt, pad at break*=1mm,colback=cellbackground, colframe=cellborder]
\prompt{In}{incolor}{73}{\boxspacing}
\begin{Verbatim}[commandchars=\\\{\}]
\PY{c+c1}{\PYZsh{} Ile osob podróżowało w danej klasie}
\PY{n}{pd}\PY{o}{.}\PY{n}{DataFrame}\PY{p}{(}\PY{n}{df}\PY{p}{[}\PY{l+s+s1}{\PYZsq{}}\PY{l+s+s1}{pclass}\PY{l+s+s1}{\PYZsq{}}\PY{p}{]}\PY{o}{.}\PY{n}{value\PYZus{}counts}\PY{p}{(}\PY{p}{)}\PY{o}{.}\PY{n}{sort\PYZus{}index}\PY{p}{(}\PY{p}{)}\PY{p}{)}
\end{Verbatim}
\end{tcolorbox}

            \begin{tcolorbox}[breakable, size=fbox, boxrule=.5pt, pad at break*=1mm, opacityfill=0]
\prompt{Out}{outcolor}{73}{\boxspacing}
\begin{Verbatim}[commandchars=\\\{\}]
        count
pclass
1.0       323
2.0       277
3.0       709
\end{Verbatim}
\end{tcolorbox}
        
    \begin{tcolorbox}[breakable, size=fbox, boxrule=1pt, pad at break*=1mm,colback=cellbackground, colframe=cellborder]
\prompt{In}{incolor}{74}{\boxspacing}
\begin{Verbatim}[commandchars=\\\{\}]
\PY{n}{pcl} \PY{o}{=} \PY{n}{df}\PY{p}{[}\PY{l+s+s1}{\PYZsq{}}\PY{l+s+s1}{pclass}\PY{l+s+s1}{\PYZsq{}}\PY{p}{]}\PY{o}{.}\PY{n}{value\PYZus{}counts}\PY{p}{(}\PY{p}{)}\PY{o}{.}\PY{n}{sort\PYZus{}index}\PY{p}{(}\PY{p}{)}
\PY{n}{plt}\PY{o}{.}\PY{n}{figure}\PY{p}{(}\PY{n}{figsize}\PY{o}{=}\PY{p}{(}\PY{l+m+mi}{6}\PY{p}{,} \PY{l+m+mi}{6}\PY{p}{)}\PY{p}{)}
\PY{n}{plt}\PY{o}{.}\PY{n}{pie}\PY{p}{(}\PY{n}{pcl}\PY{p}{,} \PY{n}{labels}\PY{o}{=}\PY{n}{pcl}\PY{o}{.}\PY{n}{index}\PY{p}{,} \PY{n}{autopct}\PY{o}{=}\PY{k}{lambda} \PY{n}{p}\PY{p}{:} \PY{l+s+sa}{f}\PY{l+s+s1}{\PYZsq{}}\PY{l+s+si}{\PYZob{}}\PY{n+nb}{int}\PY{p}{(}\PY{n}{p}\PY{+w}{ }\PY{o}{*}\PY{+w}{ }\PY{n+nb}{sum}\PY{p}{(}\PY{n}{pcl}\PY{p}{)}\PY{+w}{ }\PY{o}{/}\PY{+w}{ }\PY{l+m+mi}{100}\PY{p}{)}\PY{l+s+si}{\PYZcb{}}\PY{l+s+s1}{\PYZsq{}}\PY{p}{)}
\PY{n}{plt}\PY{o}{.}\PY{n}{title}\PY{p}{(}\PY{l+s+s1}{\PYZsq{}}\PY{l+s+s1}{Ilość pasażerów w danej klasie}\PY{l+s+s1}{\PYZsq{}}\PY{p}{)}\PY{p}{;}
\end{Verbatim}
\end{tcolorbox}

    \begin{center}
    \adjustimage{max size={0.9\linewidth}{0.9\paperheight}}{titanic_prezentacja_files/titanic_prezentacja_27_0.png}
    \end{center}
    { \hspace*{\fill} \\}
    
    \subparagraph{Mamy tutaj 3 klasy, w których podróżowali
pasażerowie.}\label{mamy-tutaj-3-klasy-w-ktuxf3rych-podruxf3ux17cowali-pasaux17cerowie.}

\subparagraph{W klasie 1 podróżowało 323 pasażerów, w klasie 2
podróżowało 277 pasażerów, w klasie 3 podróżowało 709
pasażerów.}\label{w-klasie-1-podruxf3ux17cowaux142o-323-pasaux17ceruxf3w-w-klasie-2-podruxf3ux17cowaux142o-277-pasaux17ceruxf3w-w-klasie-3-podruxf3ux17cowaux142o-709-pasaux17ceruxf3w.}

    \paragraph{SURVIVED - ilość ocalałych i
ofiar}\label{survived---iloux15bux107-ocalaux142ych-i-ofiar}

    \begin{tcolorbox}[breakable, size=fbox, boxrule=1pt, pad at break*=1mm,colback=cellbackground, colframe=cellborder]
\prompt{In}{incolor}{75}{\boxspacing}
\begin{Verbatim}[commandchars=\\\{\}]
\PY{c+c1}{\PYZsh{} Ile osób zginęło/przeżyło}
\PY{n}{passengers} \PY{o}{=} \PY{n}{df}\PY{o}{.}\PY{n}{groupby}\PY{p}{(}\PY{p}{[}\PY{l+s+s1}{\PYZsq{}}\PY{l+s+s1}{survived}\PY{l+s+s1}{\PYZsq{}}\PY{p}{]}\PY{p}{)}\PY{o}{.}\PY{n}{size}\PY{p}{(}\PY{p}{)}
\PY{n}{passengers}\PY{o}{.}\PY{n}{index} \PY{o}{=} \PY{p}{[}\PY{l+s+s1}{\PYZsq{}}\PY{l+s+s1}{Zginęło}\PY{l+s+s1}{\PYZsq{}}\PY{p}{,} \PY{l+s+s1}{\PYZsq{}}\PY{l+s+s1}{Przeżyło}\PY{l+s+s1}{\PYZsq{}}\PY{p}{]}
\PY{n}{pd}\PY{o}{.}\PY{n}{DataFrame}\PY{p}{(}\PY{n}{passengers}\PY{p}{,} \PY{n}{columns} \PY{o}{=} \PY{p}{[}\PY{l+s+s1}{\PYZsq{}}\PY{l+s+s1}{Ilość}\PY{l+s+s1}{\PYZsq{}}\PY{p}{]}\PY{p}{)}
\end{Verbatim}
\end{tcolorbox}

            \begin{tcolorbox}[breakable, size=fbox, boxrule=.5pt, pad at break*=1mm, opacityfill=0]
\prompt{Out}{outcolor}{75}{\boxspacing}
\begin{Verbatim}[commandchars=\\\{\}]
          Ilość
Zginęło     809
Przeżyło    500
\end{Verbatim}
\end{tcolorbox}
        
    \subparagraph{Katasrtofę przeżyło 500 pasażerów, zginęło 809
pasażerów.}\label{katasrtofux119-przeux17cyux142o-500-pasaux17ceruxf3w-zginux119ux142o-809-pasaux17ceruxf3w.}

    \paragraph{SEX - ilość kobiet i mężczyzn wśród pasażerów, dane o
ofiarach}\label{sex---iloux15bux107-kobiet-i-mux119ux17cczyzn-wux15bruxf3d-pasaux17ceruxf3w-dane-o-ofiarach}

    \begin{tcolorbox}[breakable, size=fbox, boxrule=1pt, pad at break*=1mm,colback=cellbackground, colframe=cellborder]
\prompt{In}{incolor}{76}{\boxspacing}
\begin{Verbatim}[commandchars=\\\{\}]
\PY{c+c1}{\PYZsh{} Ile kobiet, mężczyzn podróżowało}
\PY{n}{journey\PYZus{}m\PYZus{}f} \PY{o}{=} \PY{n}{df}\PY{o}{.}\PY{n}{groupby}\PY{p}{(}\PY{p}{[}\PY{l+s+s1}{\PYZsq{}}\PY{l+s+s1}{sex}\PY{l+s+s1}{\PYZsq{}}\PY{p}{]}\PY{p}{)}\PY{o}{.}\PY{n}{size}\PY{p}{(}\PY{p}{)}
\PY{n}{journey\PYZus{}m\PYZus{}f}\PY{o}{.}\PY{n}{index} \PY{o}{=} \PY{p}{[}\PY{l+s+s1}{\PYZsq{}}\PY{l+s+s1}{Kobiet}\PY{l+s+s1}{\PYZsq{}}\PY{p}{,} \PY{l+s+s1}{\PYZsq{}}\PY{l+s+s1}{Mężczyzn}\PY{l+s+s1}{\PYZsq{}}\PY{p}{]}
\PY{n}{pd}\PY{o}{.}\PY{n}{DataFrame}\PY{p}{(}\PY{n}{journey\PYZus{}m\PYZus{}f}\PY{p}{,} \PY{n}{columns} \PY{o}{=} \PY{p}{[}\PY{l+s+s1}{\PYZsq{}}\PY{l+s+s1}{Ilość}\PY{l+s+s1}{\PYZsq{}}\PY{p}{]}\PY{p}{)}
\end{Verbatim}
\end{tcolorbox}

            \begin{tcolorbox}[breakable, size=fbox, boxrule=.5pt, pad at break*=1mm, opacityfill=0]
\prompt{Out}{outcolor}{76}{\boxspacing}
\begin{Verbatim}[commandchars=\\\{\}]
          Ilość
Kobiet      466
Mężczyzn    843
\end{Verbatim}
\end{tcolorbox}
        
    \subparagraph{Wśród pasażerów było 466 i 843
mężczyzn}\label{wux15bruxf3d-pasaux17ceruxf3w-byux142o-466-i-843-mux119ux17cczyzn}

    \begin{tcolorbox}[breakable, size=fbox, boxrule=1pt, pad at break*=1mm,colback=cellbackground, colframe=cellborder]
\prompt{In}{incolor}{77}{\boxspacing}
\begin{Verbatim}[commandchars=\\\{\}]
\PY{c+c1}{\PYZsh{} Ile kobiet, mężczyzn zginęło/przeżyło}
\PY{n}{survived\PYZus{}m\PYZus{}f} \PY{o}{=} \PY{n}{df}\PY{o}{.}\PY{n}{groupby}\PY{p}{(}\PY{p}{[}\PY{l+s+s1}{\PYZsq{}}\PY{l+s+s1}{survived}\PY{l+s+s1}{\PYZsq{}}\PY{p}{,} \PY{l+s+s1}{\PYZsq{}}\PY{l+s+s1}{sex}\PY{l+s+s1}{\PYZsq{}}\PY{p}{]}\PY{p}{)}\PY{o}{.}\PY{n}{size}\PY{p}{(}\PY{p}{)}\PY{o}{.}\PY{n}{unstack}\PY{p}{(}\PY{p}{)}
\PY{n}{survived\PYZus{}m\PYZus{}f}\PY{o}{.}\PY{n}{index} \PY{o}{=} \PY{p}{[}\PY{l+s+s1}{\PYZsq{}}\PY{l+s+s1}{Zginęło}\PY{l+s+s1}{\PYZsq{}}\PY{p}{,} \PY{l+s+s1}{\PYZsq{}}\PY{l+s+s1}{Przeżyło}\PY{l+s+s1}{\PYZsq{}}\PY{p}{]}
\PY{n}{survived\PYZus{}m\PYZus{}f}\PY{o}{.}\PY{n}{columns} \PY{o}{=} \PY{p}{[}\PY{l+s+s1}{\PYZsq{}}\PY{l+s+s1}{Kobiet}\PY{l+s+s1}{\PYZsq{}}\PY{p}{,} \PY{l+s+s1}{\PYZsq{}}\PY{l+s+s1}{Mężczyzn}\PY{l+s+s1}{\PYZsq{}}\PY{p}{]}
\PY{n}{survived\PYZus{}m\PYZus{}f}
\end{Verbatim}
\end{tcolorbox}

            \begin{tcolorbox}[breakable, size=fbox, boxrule=.5pt, pad at break*=1mm, opacityfill=0]
\prompt{Out}{outcolor}{77}{\boxspacing}
\begin{Verbatim}[commandchars=\\\{\}]
          Kobiet  Mężczyzn
Zginęło      127       682
Przeżyło     339       161
\end{Verbatim}
\end{tcolorbox}
        
    \subparagraph{Spośród 500 ocalonych, przeżyło 339 kobiet i 161
meżczyzn.}\label{spoux15bruxf3d-500-ocalonych-przeux17cyux142o-339-kobiet-i-161-meux17cczyzn.}

    \begin{tcolorbox}[breakable, size=fbox, boxrule=1pt, pad at break*=1mm,colback=cellbackground, colframe=cellborder]
\prompt{In}{incolor}{78}{\boxspacing}
\begin{Verbatim}[commandchars=\\\{\}]
\PY{n}{colors} \PY{o}{=} \PY{p}{\PYZob{}}\PY{l+s+s1}{\PYZsq{}}\PY{l+s+s1}{Mężczyzn}\PY{l+s+s1}{\PYZsq{}}\PY{p}{:} \PY{l+s+s1}{\PYZsq{}}\PY{l+s+s1}{skyblue}\PY{l+s+s1}{\PYZsq{}}\PY{p}{,} \PY{l+s+s1}{\PYZsq{}}\PY{l+s+s1}{Kobiet}\PY{l+s+s1}{\PYZsq{}}\PY{p}{:} \PY{l+s+s1}{\PYZsq{}}\PY{l+s+s1}{orchid}\PY{l+s+s1}{\PYZsq{}}\PY{p}{\PYZcb{}}
\PY{n}{fig}\PY{p}{,} \PY{n}{axes} \PY{o}{=} \PY{n}{plt}\PY{o}{.}\PY{n}{subplots}\PY{p}{(}\PY{l+m+mi}{1}\PY{p}{,} \PY{l+m+mi}{2}\PY{p}{,} \PY{n}{figsize}\PY{o}{=}\PY{p}{(}\PY{l+m+mi}{10}\PY{p}{,} \PY{l+m+mi}{4}\PY{p}{)}\PY{p}{)}
\PY{n}{fig}\PY{o}{.}\PY{n}{suptitle}\PY{p}{(}\PY{l+s+s1}{\PYZsq{}}\PY{l+s+s1}{Procentowy udział Kobiet i Mężczyzn wśród ofiar}\PY{l+s+s1}{\PYZsq{}}\PY{p}{,} \PY{n}{fontsize}\PY{o}{=}\PY{l+m+mi}{16}\PY{p}{)}

\PY{c+c1}{\PYZsh{} Wykres Zginęło}
\PY{n}{axes}\PY{p}{[}\PY{l+m+mi}{0}\PY{p}{]}\PY{o}{.}\PY{n}{pie}\PY{p}{(}\PY{n}{survived\PYZus{}m\PYZus{}f}\PY{o}{.}\PY{n}{loc}\PY{p}{[}\PY{l+s+s1}{\PYZsq{}}\PY{l+s+s1}{Zginęło}\PY{l+s+s1}{\PYZsq{}}\PY{p}{]}\PY{p}{,} \PY{n}{labels}\PY{o}{=}\PY{n}{survived\PYZus{}m\PYZus{}f}\PY{o}{.}\PY{n}{columns}\PY{p}{,} \PY{n}{autopct}\PY{o}{=}\PY{l+s+s1}{\PYZsq{}}\PY{l+s+si}{\PYZpc{}1.1f}\PY{l+s+si}{\PYZpc{}\PYZpc{}}\PY{l+s+s1}{\PYZsq{}}\PY{p}{,} 
            \PY{n}{colors}\PY{o}{=}\PY{p}{[}\PY{n}{colors}\PY{p}{[}\PY{n}{sex}\PY{p}{]} \PY{k}{for} \PY{n}{sex} \PY{o+ow}{in} \PY{n}{survived\PYZus{}m\PYZus{}f}\PY{o}{.}\PY{n}{columns}\PY{p}{]}\PY{p}{,} \PY{n}{startangle}\PY{o}{=}\PY{l+m+mi}{140}\PY{p}{)}

\PY{c+c1}{\PYZsh{} Wykres Przeżyło}
\PY{n}{axes}\PY{p}{[}\PY{l+m+mi}{1}\PY{p}{]}\PY{o}{.}\PY{n}{pie}\PY{p}{(}\PY{n}{survived\PYZus{}m\PYZus{}f}\PY{o}{.}\PY{n}{loc}\PY{p}{[}\PY{l+s+s1}{\PYZsq{}}\PY{l+s+s1}{Przeżyło}\PY{l+s+s1}{\PYZsq{}}\PY{p}{]}\PY{p}{,} \PY{n}{labels}\PY{o}{=}\PY{n}{survived\PYZus{}m\PYZus{}f}\PY{o}{.}\PY{n}{columns}\PY{p}{,} \PY{n}{autopct}\PY{o}{=}\PY{l+s+s1}{\PYZsq{}}\PY{l+s+si}{\PYZpc{}1.1f}\PY{l+s+si}{\PYZpc{}\PYZpc{}}\PY{l+s+s1}{\PYZsq{}}\PY{p}{,} 
            \PY{n}{colors}\PY{o}{=}\PY{p}{[}\PY{n}{colors}\PY{p}{[}\PY{n}{sex}\PY{p}{]} \PY{k}{for} \PY{n}{sex} \PY{o+ow}{in} \PY{n}{survived\PYZus{}m\PYZus{}f}\PY{o}{.}\PY{n}{columns}\PY{p}{]}\PY{p}{,} \PY{n}{startangle}\PY{o}{=}\PY{l+m+mi}{50}\PY{p}{)}

\PY{c+c1}{\PYZsh{} Wyświetlenie wykresu}
\PY{n}{plt}\PY{o}{.}\PY{n}{show}\PY{p}{(}\PY{p}{)}
\end{Verbatim}
\end{tcolorbox}

    \begin{center}
    \adjustimage{max size={0.9\linewidth}{0.9\paperheight}}{titanic_prezentacja_files/titanic_prezentacja_37_0.png}
    \end{center}
    { \hspace*{\fill} \\}
    
    \subparagraph{Ilość ocalonych pasażerów w każdej z
klas}\label{iloux15bux107-ocalonych-pasaux17ceruxf3w-w-kaux17cdej-z-klas}

    \begin{tcolorbox}[breakable, size=fbox, boxrule=1pt, pad at break*=1mm,colback=cellbackground, colframe=cellborder]
\prompt{In}{incolor}{79}{\boxspacing}
\begin{Verbatim}[commandchars=\\\{\}]
\PY{c+c1}{\PYZsh{} Ile osób przeżyło w danej klasie }
\PY{n}{survived\PYZus{}by\PYZus{}class} \PY{o}{=} \PY{n}{df}\PY{o}{.}\PY{n}{groupby}\PY{p}{(}\PY{l+s+s1}{\PYZsq{}}\PY{l+s+s1}{pclass}\PY{l+s+s1}{\PYZsq{}}\PY{p}{)}\PY{p}{[}\PY{l+s+s1}{\PYZsq{}}\PY{l+s+s1}{survived}\PY{l+s+s1}{\PYZsq{}}\PY{p}{]}\PY{o}{.}\PY{n}{sum}\PY{p}{(}\PY{p}{)}
\PY{n}{survived\PYZus{}by\PYZus{}class}\PY{o}{.}\PY{n}{columns} \PY{o}{=} \PY{p}{[}\PY{l+s+s1}{\PYZsq{}}\PY{l+s+s1}{Przeżyło}\PY{l+s+s1}{\PYZsq{}}\PY{p}{]}
\PY{n}{pd}\PY{o}{.}\PY{n}{DataFrame}\PY{p}{(}\PY{n}{survived\PYZus{}by\PYZus{}class}\PY{p}{)}

\PY{c+c1}{\PYZsh{}pd.DataFrame(df.groupby(\PYZsq{}pclass\PYZsq{})[\PYZsq{}survived\PYZsq{}].sum())}
\end{Verbatim}
\end{tcolorbox}

            \begin{tcolorbox}[breakable, size=fbox, boxrule=.5pt, pad at break*=1mm, opacityfill=0]
\prompt{Out}{outcolor}{79}{\boxspacing}
\begin{Verbatim}[commandchars=\\\{\}]
        survived
pclass
1.0        200.0
2.0        119.0
3.0        181.0
\end{Verbatim}
\end{tcolorbox}
        
    \begin{tcolorbox}[breakable, size=fbox, boxrule=1pt, pad at break*=1mm,colback=cellbackground, colframe=cellborder]
\prompt{In}{incolor}{80}{\boxspacing}
\begin{Verbatim}[commandchars=\\\{\}]
\PY{n}{survived\PYZus{}by\PYZus{}class\PYZus{}m\PYZus{}f} \PY{o}{=} \PY{n}{df}\PY{p}{[}\PY{n}{df}\PY{p}{[}\PY{l+s+s1}{\PYZsq{}}\PY{l+s+s1}{survived}\PY{l+s+s1}{\PYZsq{}}\PY{p}{]} \PY{o}{==} \PY{l+m+mi}{1}\PY{p}{]}\PY{o}{.}\PY{n}{groupby}\PY{p}{(}\PY{p}{[}\PY{l+s+s1}{\PYZsq{}}\PY{l+s+s1}{pclass}\PY{l+s+s1}{\PYZsq{}}\PY{p}{,} \PY{l+s+s1}{\PYZsq{}}\PY{l+s+s1}{sex}\PY{l+s+s1}{\PYZsq{}}\PY{p}{]}\PY{p}{)}\PY{o}{.}\PY{n}{size}\PY{p}{(}\PY{p}{)}\PY{o}{.}\PY{n}{unstack}\PY{p}{(}\PY{p}{)}

\PY{n}{survived\PYZus{}by\PYZus{}class\PYZus{}m\PYZus{}f}\PY{o}{.}\PY{n}{plot}\PY{p}{(}\PY{n}{kind}\PY{o}{=}\PY{l+s+s1}{\PYZsq{}}\PY{l+s+s1}{bar}\PY{l+s+s1}{\PYZsq{}}\PY{p}{,} \PY{n}{figsize}\PY{o}{=}\PY{p}{(}\PY{l+m+mi}{8}\PY{p}{,} \PY{l+m+mi}{4}\PY{p}{)}\PY{p}{,} \PY{n}{color}\PY{o}{=}\PY{p}{[}\PY{l+s+s1}{\PYZsq{}}\PY{l+s+s1}{orchid}\PY{l+s+s1}{\PYZsq{}}\PY{p}{,} \PY{l+s+s1}{\PYZsq{}}\PY{l+s+s1}{skyblue}\PY{l+s+s1}{\PYZsq{}}\PY{p}{]}\PY{p}{)}

\PY{c+c1}{\PYZsh{} Dodanie etykiet i tytułu}
\PY{n}{plt}\PY{o}{.}\PY{n}{title}\PY{p}{(}\PY{l+s+s1}{\PYZsq{}}\PY{l+s+s1}{Liczba ocalałych w każdej klasie według płci}\PY{l+s+s1}{\PYZsq{}}\PY{p}{)}
\PY{n}{plt}\PY{o}{.}\PY{n}{xlabel}\PY{p}{(}\PY{l+s+s1}{\PYZsq{}}\PY{l+s+s1}{Klasa}\PY{l+s+s1}{\PYZsq{}}\PY{p}{)}
\PY{n}{plt}\PY{o}{.}\PY{n}{ylabel}\PY{p}{(}\PY{l+s+s1}{\PYZsq{}}\PY{l+s+s1}{Liczba ocalałych}\PY{l+s+s1}{\PYZsq{}}\PY{p}{)}
\PY{n}{plt}\PY{o}{.}\PY{n}{legend}\PY{p}{(}\PY{n}{title}\PY{o}{=}\PY{l+s+s1}{\PYZsq{}}\PY{l+s+s1}{Płeć}\PY{l+s+s1}{\PYZsq{}}\PY{p}{,} \PY{n}{labels}\PY{o}{=}\PY{p}{[}\PY{l+s+s1}{\PYZsq{}}\PY{l+s+s1}{Kobiet}\PY{l+s+s1}{\PYZsq{}}\PY{p}{,} \PY{l+s+s1}{\PYZsq{}}\PY{l+s+s1}{Mężczyzn}\PY{l+s+s1}{\PYZsq{}}\PY{p}{]}\PY{p}{)}
\PY{n}{plt}\PY{o}{.}\PY{n}{xticks}\PY{p}{(}\PY{n}{rotation}\PY{o}{=}\PY{l+m+mi}{0}\PY{p}{)}  
\PY{n}{plt}\PY{o}{.}\PY{n}{show}\PY{p}{(}\PY{p}{)}
\end{Verbatim}
\end{tcolorbox}

    \begin{center}
    \adjustimage{max size={0.9\linewidth}{0.9\paperheight}}{titanic_prezentacja_files/titanic_prezentacja_40_0.png}
    \end{center}
    { \hspace*{\fill} \\}
    
    \subparagraph{Podróżujących w klasie 1 ocalało 200 z 323 osób, w klasie
2 ocalało 119 z 277 osób, w klasie 3 ocalało 181 z 709
osób}\label{podruxf3ux17cujux105cych-w-klasie-1-ocalaux142o-200-z-323-osuxf3b-w-klasie-2-ocalaux142o-119-z-277-osuxf3b-w-klasie-3-ocalaux142o-181-z-709-osuxf3b}

    \paragraph{AGE - wiek pasażerów}\label{age---wiek-pasaux17ceruxf3w}

    \begin{tcolorbox}[breakable, size=fbox, boxrule=1pt, pad at break*=1mm,colback=cellbackground, colframe=cellborder]
\prompt{In}{incolor}{81}{\boxspacing}
\begin{Verbatim}[commandchars=\\\{\}]
\PY{n}{pd}\PY{o}{.}\PY{n}{DataFrame}\PY{p}{(}\PY{n}{df}\PY{p}{[}\PY{l+s+s1}{\PYZsq{}}\PY{l+s+s1}{age}\PY{l+s+s1}{\PYZsq{}}\PY{p}{]}\PY{o}{.}\PY{n}{unique}\PY{p}{(}\PY{p}{)}\PY{p}{)}
\end{Verbatim}
\end{tcolorbox}

            \begin{tcolorbox}[breakable, size=fbox, boxrule=.5pt, pad at break*=1mm, opacityfill=0]
\prompt{Out}{outcolor}{81}{\boxspacing}
\begin{Verbatim}[commandchars=\\\{\}]
          0
0   29.0000
1    0.9167
2    2.0000
3   30.0000
4   25.0000
..      {\ldots}
94  60.5000
95  74.0000
96   0.4167
97  11.5000
98  26.5000

[99 rows x 1 columns]
\end{Verbatim}
\end{tcolorbox}
        
    \subparagraph{Ponieważ wiek nie jest podany w liczbach całkowitych,
zaokraglę go i zapiszę w nowej
kolumnie}\label{poniewaux17c-wiek-nie-jest-podany-w-liczbach-caux142kowitych-zaokraglux119-go-i-zapiszux119-w-nowej-kolumnie}

    \begin{tcolorbox}[breakable, size=fbox, boxrule=1pt, pad at break*=1mm,colback=cellbackground, colframe=cellborder]
\prompt{In}{incolor}{125}{\boxspacing}
\begin{Verbatim}[commandchars=\\\{\}]
\PY{n}{df}\PY{p}{[}\PY{l+s+s1}{\PYZsq{}}\PY{l+s+s1}{age\PYZus{}round}\PY{l+s+s1}{\PYZsq{}}\PY{p}{]} \PY{o}{=} \PY{n}{df}\PY{p}{[}\PY{l+s+s1}{\PYZsq{}}\PY{l+s+s1}{age}\PY{l+s+s1}{\PYZsq{}}\PY{p}{]}\PY{o}{.}\PY{n}{round}\PY{p}{(}\PY{p}{)}
\PY{n}{pd}\PY{o}{.}\PY{n}{DataFrame}\PY{p}{(}\PY{n}{df}\PY{p}{[}\PY{l+s+s1}{\PYZsq{}}\PY{l+s+s1}{age\PYZus{}round}\PY{l+s+s1}{\PYZsq{}}\PY{p}{]}\PY{o}{.}\PY{n}{unique}\PY{p}{(}\PY{p}{)}\PY{p}{)}
\end{Verbatim}
\end{tcolorbox}

            \begin{tcolorbox}[breakable, size=fbox, boxrule=.5pt, pad at break*=1mm, opacityfill=0]
\prompt{Out}{outcolor}{125}{\boxspacing}
\begin{Verbatim}[commandchars=\\\{\}]
       0
0   29.0
1    1.0
2    2.0
3   30.0
4   25.0
..   {\ldots}
68  66.0
69   9.0
70   0.0
71  10.0
72  74.0

[73 rows x 1 columns]
\end{Verbatim}
\end{tcolorbox}
        
    \subparagraph{Po zaokragleniu wieku pasażerów do liczb całkowitych,
otrzymałem 74 wartośći unikatowe. Dane zapisałem w nowej kolumnie -
age\_round.}\label{po-zaokragleniu-wieku-pasaux17ceruxf3w-do-liczb-caux142kowitych-otrzymaux142em-74-wartoux15bux107i-unikatowe.-dane-zapisaux142em-w-nowej-kolumnie---age_round.}

    \subparagraph{Średni wiek pasażerów to blisko 30 lat, najmłodszy pasażer
jest noworodkiem poniżej pół roku życia, najstrszy pasażer ma 80
lat.}\label{ux15bredni-wiek-pasaux17ceruxf3w-to-blisko-30-lat-najmux142odszy-pasaux17cer-jest-noworodkiem-poniux17cej-puxf3ux142-roku-ux17cycia-najstrszy-pasaux17cer-ma-80-lat.}

    \begin{tcolorbox}[breakable, size=fbox, boxrule=1pt, pad at break*=1mm,colback=cellbackground, colframe=cellborder]
\prompt{In}{incolor}{83}{\boxspacing}
\begin{Verbatim}[commandchars=\\\{\}]
\PY{n}{pd}\PY{o}{.}\PY{n}{DataFrame}\PY{p}{(}\PY{n}{df}\PY{p}{[}\PY{l+s+s1}{\PYZsq{}}\PY{l+s+s1}{age\PYZus{}round}\PY{l+s+s1}{\PYZsq{}}\PY{p}{]}\PY{o}{.}\PY{n}{describe}\PY{p}{(}\PY{p}{)}\PY{p}{)}
\end{Verbatim}
\end{tcolorbox}

            \begin{tcolorbox}[breakable, size=fbox, boxrule=.5pt, pad at break*=1mm, opacityfill=0]
\prompt{Out}{outcolor}{83}{\boxspacing}
\begin{Verbatim}[commandchars=\\\{\}]
         age\_round
count  1046.000000
mean     29.870937
std      14.411571
min       0.000000
25\%      21.000000
50\%      28.000000
75\%      39.000000
max      80.000000
\end{Verbatim}
\end{tcolorbox}
        
    \begin{tcolorbox}[breakable, size=fbox, boxrule=1pt, pad at break*=1mm,colback=cellbackground, colframe=cellborder]
\prompt{In}{incolor}{84}{\boxspacing}
\begin{Verbatim}[commandchars=\\\{\}]
\PY{c+c1}{\PYZsh{} zmiana nazwy oryginalnej kolumny \PYZdq{}age\PYZdq{}}
\PY{n}{df} \PY{o}{=} \PY{n}{df}\PY{o}{.}\PY{n}{rename}\PY{p}{(}\PY{n}{columns}\PY{o}{=}\PY{p}{\PYZob{}}\PY{l+s+s1}{\PYZsq{}}\PY{l+s+s1}{age}\PY{l+s+s1}{\PYZsq{}}\PY{p}{:} \PY{l+s+s1}{\PYZsq{}}\PY{l+s+s1}{age\PYZus{}raw}\PY{l+s+s1}{\PYZsq{}}\PY{p}{\PYZcb{}}\PY{p}{)}
\end{Verbatim}
\end{tcolorbox}

    \begin{tcolorbox}[breakable, size=fbox, boxrule=1pt, pad at break*=1mm,colback=cellbackground, colframe=cellborder]
\prompt{In}{incolor}{85}{\boxspacing}
\begin{Verbatim}[commandchars=\\\{\}]
\PY{c+c1}{\PYZsh{} zmiana kolumny \PYZdq{}age\PYZus{}round\PYZdq{}, na kolumne \PYZdq{}age\PYZdq{}}
\PY{n}{df} \PY{o}{=} \PY{n}{df}\PY{o}{.}\PY{n}{rename}\PY{p}{(}\PY{n}{columns}\PY{o}{=}\PY{p}{\PYZob{}}\PY{l+s+s1}{\PYZsq{}}\PY{l+s+s1}{age\PYZus{}round}\PY{l+s+s1}{\PYZsq{}}\PY{p}{:} \PY{l+s+s1}{\PYZsq{}}\PY{l+s+s1}{age}\PY{l+s+s1}{\PYZsq{}}\PY{p}{\PYZcb{}}\PY{p}{)}
\end{Verbatim}
\end{tcolorbox}

    \begin{tcolorbox}[breakable, size=fbox, boxrule=1pt, pad at break*=1mm,colback=cellbackground, colframe=cellborder]
\prompt{In}{incolor}{86}{\boxspacing}
\begin{Verbatim}[commandchars=\\\{\}]
\PY{c+c1}{\PYZsh{} Podział danych na kobiety i mężczyzn}
\PY{n}{female} \PY{o}{=} \PY{n}{df}\PY{p}{[}\PY{n}{df}\PY{p}{[}\PY{l+s+s1}{\PYZsq{}}\PY{l+s+s1}{sex}\PY{l+s+s1}{\PYZsq{}}\PY{p}{]} \PY{o}{==} \PY{l+s+s1}{\PYZsq{}}\PY{l+s+s1}{female}\PY{l+s+s1}{\PYZsq{}}\PY{p}{]}\PY{p}{[}\PY{l+s+s1}{\PYZsq{}}\PY{l+s+s1}{age}\PY{l+s+s1}{\PYZsq{}}\PY{p}{]}
\PY{n}{men} \PY{o}{=} \PY{n}{df}\PY{p}{[}\PY{n}{df}\PY{p}{[}\PY{l+s+s1}{\PYZsq{}}\PY{l+s+s1}{sex}\PY{l+s+s1}{\PYZsq{}}\PY{p}{]} \PY{o}{==} \PY{l+s+s1}{\PYZsq{}}\PY{l+s+s1}{male}\PY{l+s+s1}{\PYZsq{}}\PY{p}{]}\PY{p}{[}\PY{l+s+s1}{\PYZsq{}}\PY{l+s+s1}{age}\PY{l+s+s1}{\PYZsq{}}\PY{p}{]}

\PY{c+c1}{\PYZsh{} Tworzenie histogramu}
\PY{n}{plt}\PY{o}{.}\PY{n}{figure}\PY{p}{(}\PY{n}{figsize}\PY{o}{=}\PY{p}{(}\PY{l+m+mi}{10}\PY{p}{,} \PY{l+m+mi}{6}\PY{p}{)}\PY{p}{)}
\PY{n}{plt}\PY{o}{.}\PY{n}{hist}\PY{p}{(}\PY{n}{female}\PY{p}{,} \PY{n}{bins}\PY{o}{=}\PY{l+m+mi}{25}\PY{p}{,} \PY{n}{alpha}\PY{o}{=}\PY{l+m+mf}{0.8}\PY{p}{,} \PY{n}{color}\PY{o}{=}\PY{l+s+s1}{\PYZsq{}}\PY{l+s+s1}{orchid}\PY{l+s+s1}{\PYZsq{}}\PY{p}{,} \PY{n}{label}\PY{o}{=}\PY{l+s+s1}{\PYZsq{}}\PY{l+s+s1}{Kobiety}\PY{l+s+s1}{\PYZsq{}}\PY{p}{,} \PY{n}{edgecolor}\PY{o}{=}\PY{l+s+s1}{\PYZsq{}}\PY{l+s+s1}{black}\PY{l+s+s1}{\PYZsq{}}\PY{p}{)}
\PY{n}{plt}\PY{o}{.}\PY{n}{hist}\PY{p}{(}\PY{n}{men}\PY{p}{,} \PY{n}{bins}\PY{o}{=}\PY{l+m+mi}{25}\PY{p}{,} \PY{n}{alpha}\PY{o}{=}\PY{l+m+mf}{0.4}\PY{p}{,} \PY{n}{color}\PY{o}{=}\PY{l+s+s1}{\PYZsq{}}\PY{l+s+s1}{skyblue}\PY{l+s+s1}{\PYZsq{}}\PY{p}{,} \PY{n}{label}\PY{o}{=}\PY{l+s+s1}{\PYZsq{}}\PY{l+s+s1}{Mężczyźni}\PY{l+s+s1}{\PYZsq{}}\PY{p}{,} \PY{n}{edgecolor}\PY{o}{=}\PY{l+s+s1}{\PYZsq{}}\PY{l+s+s1}{black}\PY{l+s+s1}{\PYZsq{}}\PY{p}{)}

\PY{c+c1}{\PYZsh{} Ustawienia wykresu}
\PY{n}{plt}\PY{o}{.}\PY{n}{title}\PY{p}{(}\PY{l+s+s1}{\PYZsq{}}\PY{l+s+s1}{Rozkład wieku pasażerów z podziałem na płeć}\PY{l+s+s1}{\PYZsq{}}\PY{p}{)}
\PY{n}{plt}\PY{o}{.}\PY{n}{xlabel}\PY{p}{(}\PY{l+s+s1}{\PYZsq{}}\PY{l+s+s1}{Wiek}\PY{l+s+s1}{\PYZsq{}}\PY{p}{)}
\PY{n}{plt}\PY{o}{.}\PY{n}{ylabel}\PY{p}{(}\PY{l+s+s1}{\PYZsq{}}\PY{l+s+s1}{Liczba pasażerów}\PY{l+s+s1}{\PYZsq{}}\PY{p}{)}
\PY{n}{plt}\PY{o}{.}\PY{n}{legend}\PY{p}{(}\PY{p}{)}
\PY{n}{plt}\PY{o}{.}\PY{n}{grid}\PY{p}{(}\PY{n}{axis}\PY{o}{=}\PY{l+s+s1}{\PYZsq{}}\PY{l+s+s1}{y}\PY{l+s+s1}{\PYZsq{}}\PY{p}{,} \PY{n}{linestyle}\PY{o}{=}\PY{l+s+s1}{\PYZsq{}}\PY{l+s+s1}{\PYZhy{}\PYZhy{}}\PY{l+s+s1}{\PYZsq{}}\PY{p}{,} \PY{n}{alpha}\PY{o}{=}\PY{l+m+mf}{0.7}\PY{p}{)}

\PY{c+c1}{\PYZsh{} Wyświetlenie wykresu}
\PY{n}{plt}\PY{o}{.}\PY{n}{show}\PY{p}{(}\PY{p}{)}
\end{Verbatim}
\end{tcolorbox}

    \begin{center}
    \adjustimage{max size={0.9\linewidth}{0.9\paperheight}}{titanic_prezentacja_files/titanic_prezentacja_51_0.png}
    \end{center}
    { \hspace*{\fill} \\}
    
    \paragraph{SIBSP - liczba rodzeństwa, małżonków na
pokładzie}\label{sibsp---liczba-rodzeux144stwa-maux142ux17conkuxf3w-na-pokux142adzie}

    \begin{tcolorbox}[breakable, size=fbox, boxrule=1pt, pad at break*=1mm,colback=cellbackground, colframe=cellborder]
\prompt{In}{incolor}{132}{\boxspacing}
\begin{Verbatim}[commandchars=\\\{\}]
\PY{n}{sibsp\PYZus{}greater\PYZus{}than\PYZus{}zero\PYZus{}count} \PY{o}{=} \PY{n}{df}\PY{p}{[}\PY{l+s+s1}{\PYZsq{}}\PY{l+s+s1}{sibsp}\PY{l+s+s1}{\PYZsq{}}\PY{p}{]}\PY{o}{.}\PY{n}{dropna}\PY{p}{(}\PY{p}{)}\PY{o}{.}\PY{n}{gt}\PY{p}{(}\PY{l+m+mi}{0}\PY{p}{)}\PY{o}{.}\PY{n}{sum}\PY{p}{(}\PY{p}{)}
\PY{n+nb}{print}\PY{p}{(}\PY{l+s+sa}{f}\PY{l+s+s2}{\PYZdq{}}\PY{l+s+si}{\PYZob{}}\PY{n}{sibsp\PYZus{}greater\PYZus{}than\PYZus{}zero\PYZus{}count}\PY{l+s+si}{\PYZcb{}}\PY{l+s+s2}{ pasażerów było na pokładzie z rodzeństwem lub małżonkiem.}\PY{l+s+s2}{\PYZdq{}}\PY{p}{)}
\end{Verbatim}
\end{tcolorbox}

    \begin{Verbatim}[commandchars=\\\{\}]
418 pasażerów było na pokładzie z rodzeństwem lub małżonkiem.
    \end{Verbatim}

    \paragraph{PARCH - liczba rodziców, dzieci na
pokładzie}\label{parch---liczba-rodzicuxf3w-dzieci-na-pokux142adzie}

    \begin{tcolorbox}[breakable, size=fbox, boxrule=1pt, pad at break*=1mm,colback=cellbackground, colframe=cellborder]
\prompt{In}{incolor}{133}{\boxspacing}
\begin{Verbatim}[commandchars=\\\{\}]
\PY{n}{parch\PYZus{}greater\PYZus{}than\PYZus{}zero\PYZus{}count} \PY{o}{=} \PY{n}{df}\PY{p}{[}\PY{l+s+s1}{\PYZsq{}}\PY{l+s+s1}{parch}\PY{l+s+s1}{\PYZsq{}}\PY{p}{]}\PY{o}{.}\PY{n}{dropna}\PY{p}{(}\PY{p}{)}\PY{o}{.}\PY{n}{gt}\PY{p}{(}\PY{l+m+mi}{0}\PY{p}{)}\PY{o}{.}\PY{n}{sum}\PY{p}{(}\PY{p}{)}
\PY{n+nb}{print}\PY{p}{(}\PY{l+s+sa}{f}\PY{l+s+s2}{\PYZdq{}}\PY{l+s+si}{\PYZob{}}\PY{n}{parch\PYZus{}greater\PYZus{}than\PYZus{}zero\PYZus{}count}\PY{l+s+si}{\PYZcb{}}\PY{l+s+s2}{ pasażerów było na pokładzie z rodzicem lub dzieckiem.}\PY{l+s+s2}{\PYZdq{}}\PY{p}{)}
\end{Verbatim}
\end{tcolorbox}

    \begin{Verbatim}[commandchars=\\\{\}]
307 pasażerów było na pokładzie z rodzicem lub dzieckiem.
    \end{Verbatim}

    \paragraph{TICKET - numer biletu}\label{ticket---numer-biletu}

    \begin{tcolorbox}[breakable, size=fbox, boxrule=1pt, pad at break*=1mm,colback=cellbackground, colframe=cellborder]
\prompt{In}{incolor}{91}{\boxspacing}
\begin{Verbatim}[commandchars=\\\{\}]
\PY{n}{pd}\PY{o}{.}\PY{n}{DataFrame}\PY{p}{(}\PY{n}{df}\PY{p}{[}\PY{l+s+s1}{\PYZsq{}}\PY{l+s+s1}{ticket}\PY{l+s+s1}{\PYZsq{}}\PY{p}{]}\PY{o}{.}\PY{n}{describe}\PY{p}{(}\PY{p}{)}\PY{p}{)}
\end{Verbatim}
\end{tcolorbox}

            \begin{tcolorbox}[breakable, size=fbox, boxrule=.5pt, pad at break*=1mm, opacityfill=0]
\prompt{Out}{outcolor}{91}{\boxspacing}
\begin{Verbatim}[commandchars=\\\{\}]
          ticket
count       1309
unique       929
top     CA. 2343
freq          11
\end{Verbatim}
\end{tcolorbox}
        
    \subparagraph{Numery biletu mają 929 wartości unikatowych, na 1309
pozycji, należy sprawdzić
duplikaty.}\label{numery-biletu-majux105-929-wartoux15bci-unikatowych-na-1309-pozycji-naleux17cy-sprawdziux107-duplikaty.}

    \paragraph{FARE - cena biletu}\label{fare---cena-biletu}

    \begin{tcolorbox}[breakable, size=fbox, boxrule=1pt, pad at break*=1mm,colback=cellbackground, colframe=cellborder]
\prompt{In}{incolor}{92}{\boxspacing}
\begin{Verbatim}[commandchars=\\\{\}]
\PY{n}{fare\PYZus{}stats} \PY{o}{=} \PY{n}{df}\PY{o}{.}\PY{n}{groupby}\PY{p}{(}\PY{p}{[}\PY{l+s+s1}{\PYZsq{}}\PY{l+s+s1}{pclass}\PY{l+s+s1}{\PYZsq{}}\PY{p}{]}\PY{p}{)}\PY{p}{[}\PY{l+s+s1}{\PYZsq{}}\PY{l+s+s1}{fare}\PY{l+s+s1}{\PYZsq{}}\PY{p}{]}\PY{o}{.}\PY{n}{describe}\PY{p}{(}\PY{p}{)}
\PY{n}{fare\PYZus{}stats}
\end{Verbatim}
\end{tcolorbox}

            \begin{tcolorbox}[breakable, size=fbox, boxrule=.5pt, pad at break*=1mm, opacityfill=0]
\prompt{Out}{outcolor}{92}{\boxspacing}
\begin{Verbatim}[commandchars=\\\{\}]
        count       mean        std  min      25\%      50\%       75\%       max
pclass
1.0     323.0  87.508992  80.447178  0.0  30.6958  60.0000  107.6625  512.3292
2.0     277.0  21.179196  13.607122  0.0  13.0000  15.0458   26.0000   73.5000
3.0     708.0  13.302889  11.494358  0.0   7.7500   8.0500   15.2458   69.5500
\end{Verbatim}
\end{tcolorbox}
        
    \subparagraph{Cena biletu uzależniona była od klasy biletu. Średnia cena
biletu dla klasy 1 to 87, dla klasy 2 to 13, dla klasy 3 to
11.}\label{cena-biletu-uzaleux17cniona-byux142a-od-klasy-biletu.-ux15brednia-cena-biletu-dla-klasy-1-to-87-dla-klasy-2-to-13-dla-klasy-3-to-11.}

\subparagraph{Najdroższy bilet miał cenę 512, najtańsze
0.}\label{najdroux17cszy-bilet-miaux142-cenux119-512-najtaux144sze-0.}

    \begin{tcolorbox}[breakable, size=fbox, boxrule=1pt, pad at break*=1mm,colback=cellbackground, colframe=cellborder]
\prompt{In}{incolor}{93}{\boxspacing}
\begin{Verbatim}[commandchars=\\\{\}]
\PY{n}{class\PYZus{}1} \PY{o}{=} \PY{n}{df}\PY{p}{[}\PY{n}{df}\PY{p}{[}\PY{l+s+s1}{\PYZsq{}}\PY{l+s+s1}{pclass}\PY{l+s+s1}{\PYZsq{}}\PY{p}{]} \PY{o}{==} \PY{l+m+mi}{1}\PY{p}{]}\PY{p}{[}\PY{l+s+s1}{\PYZsq{}}\PY{l+s+s1}{fare}\PY{l+s+s1}{\PYZsq{}}\PY{p}{]}
\PY{n}{class\PYZus{}2} \PY{o}{=} \PY{n}{df}\PY{p}{[}\PY{n}{df}\PY{p}{[}\PY{l+s+s1}{\PYZsq{}}\PY{l+s+s1}{pclass}\PY{l+s+s1}{\PYZsq{}}\PY{p}{]} \PY{o}{==} \PY{l+m+mi}{2}\PY{p}{]}\PY{p}{[}\PY{l+s+s1}{\PYZsq{}}\PY{l+s+s1}{fare}\PY{l+s+s1}{\PYZsq{}}\PY{p}{]}
\PY{n}{class\PYZus{}3} \PY{o}{=} \PY{n}{df}\PY{p}{[}\PY{n}{df}\PY{p}{[}\PY{l+s+s1}{\PYZsq{}}\PY{l+s+s1}{pclass}\PY{l+s+s1}{\PYZsq{}}\PY{p}{]} \PY{o}{==} \PY{l+m+mi}{3}\PY{p}{]}\PY{p}{[}\PY{l+s+s1}{\PYZsq{}}\PY{l+s+s1}{fare}\PY{l+s+s1}{\PYZsq{}}\PY{p}{]}

\PY{c+c1}{\PYZsh{} Tworzenie histogramu}
\PY{n}{plt}\PY{o}{.}\PY{n}{figure}\PY{p}{(}\PY{n}{figsize}\PY{o}{=}\PY{p}{(}\PY{l+m+mi}{10}\PY{p}{,} \PY{l+m+mi}{3}\PY{p}{)}\PY{p}{)}
\PY{n}{plt}\PY{o}{.}\PY{n}{hist}\PY{p}{(}\PY{n}{class\PYZus{}1}\PY{p}{,} \PY{n}{bins}\PY{o}{=}\PY{l+m+mi}{30}\PY{p}{,} \PY{n}{alpha}\PY{o}{=}\PY{l+m+mf}{0.7}\PY{p}{,} \PY{n}{color}\PY{o}{=}\PY{l+s+s1}{\PYZsq{}}\PY{l+s+s1}{gold}\PY{l+s+s1}{\PYZsq{}}\PY{p}{,} \PY{n}{label}\PY{o}{=}\PY{l+s+s1}{\PYZsq{}}\PY{l+s+s1}{1 klasa}\PY{l+s+s1}{\PYZsq{}}\PY{p}{,} \PY{n}{edgecolor}\PY{o}{=}\PY{l+s+s1}{\PYZsq{}}\PY{l+s+s1}{black}\PY{l+s+s1}{\PYZsq{}}\PY{p}{)}
\PY{n}{plt}\PY{o}{.}\PY{n}{hist}\PY{p}{(}\PY{n}{class\PYZus{}2}\PY{p}{,} \PY{n}{bins}\PY{o}{=}\PY{l+m+mi}{30}\PY{p}{,} \PY{n}{alpha}\PY{o}{=}\PY{l+m+mf}{0.7}\PY{p}{,} \PY{n}{color}\PY{o}{=}\PY{l+s+s1}{\PYZsq{}}\PY{l+s+s1}{silver}\PY{l+s+s1}{\PYZsq{}}\PY{p}{,} \PY{n}{label}\PY{o}{=}\PY{l+s+s1}{\PYZsq{}}\PY{l+s+s1}{2 klasa}\PY{l+s+s1}{\PYZsq{}}\PY{p}{,} \PY{n}{edgecolor}\PY{o}{=}\PY{l+s+s1}{\PYZsq{}}\PY{l+s+s1}{black}\PY{l+s+s1}{\PYZsq{}}\PY{p}{)}
\PY{n}{plt}\PY{o}{.}\PY{n}{hist}\PY{p}{(}\PY{n}{class\PYZus{}3}\PY{p}{,} \PY{n}{bins}\PY{o}{=}\PY{l+m+mi}{30}\PY{p}{,} \PY{n}{alpha}\PY{o}{=}\PY{l+m+mf}{0.7}\PY{p}{,} \PY{n}{color}\PY{o}{=}\PY{l+s+s1}{\PYZsq{}}\PY{l+s+s1}{brown}\PY{l+s+s1}{\PYZsq{}}\PY{p}{,} \PY{n}{label}\PY{o}{=}\PY{l+s+s1}{\PYZsq{}}\PY{l+s+s1}{3 klasa}\PY{l+s+s1}{\PYZsq{}}\PY{p}{,} \PY{n}{edgecolor}\PY{o}{=}\PY{l+s+s1}{\PYZsq{}}\PY{l+s+s1}{black}\PY{l+s+s1}{\PYZsq{}}\PY{p}{)}

\PY{c+c1}{\PYZsh{} Ustawienia wykresu}
\PY{n}{plt}\PY{o}{.}\PY{n}{title}\PY{p}{(}\PY{l+s+s1}{\PYZsq{}}\PY{l+s+s1}{Rozkład cen biletów w zależności od klasy biletu}\PY{l+s+s1}{\PYZsq{}}\PY{p}{)}
\PY{n}{plt}\PY{o}{.}\PY{n}{xlabel}\PY{p}{(}\PY{l+s+s1}{\PYZsq{}}\PY{l+s+s1}{Cena biletu}\PY{l+s+s1}{\PYZsq{}}\PY{p}{)}
\PY{n}{plt}\PY{o}{.}\PY{n}{ylabel}\PY{p}{(}\PY{l+s+s1}{\PYZsq{}}\PY{l+s+s1}{Liczba pasażerów}\PY{l+s+s1}{\PYZsq{}}\PY{p}{)}
\PY{n}{plt}\PY{o}{.}\PY{n}{legend}\PY{p}{(}\PY{p}{)}
\PY{n}{plt}\PY{o}{.}\PY{n}{grid}\PY{p}{(}\PY{n}{axis}\PY{o}{=}\PY{l+s+s1}{\PYZsq{}}\PY{l+s+s1}{y}\PY{l+s+s1}{\PYZsq{}}\PY{p}{,} \PY{n}{linestyle}\PY{o}{=}\PY{l+s+s1}{\PYZsq{}}\PY{l+s+s1}{\PYZhy{}\PYZhy{}}\PY{l+s+s1}{\PYZsq{}}\PY{p}{,} \PY{n}{alpha}\PY{o}{=}\PY{l+m+mf}{0.7}\PY{p}{)}

\PY{c+c1}{\PYZsh{} Wyświetlenie wykresu}
\PY{n}{plt}\PY{o}{.}\PY{n}{show}\PY{p}{(}\PY{p}{)}
\end{Verbatim}
\end{tcolorbox}

    \begin{center}
    \adjustimage{max size={0.9\linewidth}{0.9\paperheight}}{titanic_prezentacja_files/titanic_prezentacja_62_0.png}
    \end{center}
    { \hspace*{\fill} \\}
    
    \subparagraph{Bardzo duża rozpiętość cen biletów, zwlaszcza w klasie 1.
Najwięcej wartośći zarejestrowanych w okolicy 10 dla klasy
3.}\label{bardzo-duux17ca-rozpiux119toux15bux107-cen-biletuxf3w-zwlaszcza-w-klasie-1.-najwiux119cej-wartoux15bux107i-zarejestrowanych-w-okolicy-10-dla-klasy-3.}

    \subparagraph{Dla lepszego zobrazowania dla klas 1 i 2, wykres z
ograniczonym
zakresem}\label{dla-lepszego-zobrazowania-dla-klas-1-i-2-wykres-z-ograniczonym-zakresem}

    \begin{tcolorbox}[breakable, size=fbox, boxrule=1pt, pad at break*=1mm,colback=cellbackground, colframe=cellborder]
\prompt{In}{incolor}{94}{\boxspacing}
\begin{Verbatim}[commandchars=\\\{\}]
\PY{n}{class\PYZus{}1} \PY{o}{=} \PY{n}{df}\PY{p}{[}\PY{n}{df}\PY{p}{[}\PY{l+s+s1}{\PYZsq{}}\PY{l+s+s1}{pclass}\PY{l+s+s1}{\PYZsq{}}\PY{p}{]} \PY{o}{==} \PY{l+m+mi}{1}\PY{p}{]}\PY{p}{[}\PY{l+s+s1}{\PYZsq{}}\PY{l+s+s1}{fare}\PY{l+s+s1}{\PYZsq{}}\PY{p}{]}
\PY{n}{class\PYZus{}2} \PY{o}{=} \PY{n}{df}\PY{p}{[}\PY{n}{df}\PY{p}{[}\PY{l+s+s1}{\PYZsq{}}\PY{l+s+s1}{pclass}\PY{l+s+s1}{\PYZsq{}}\PY{p}{]} \PY{o}{==} \PY{l+m+mi}{2}\PY{p}{]}\PY{p}{[}\PY{l+s+s1}{\PYZsq{}}\PY{l+s+s1}{fare}\PY{l+s+s1}{\PYZsq{}}\PY{p}{]}
\PY{n}{class\PYZus{}3} \PY{o}{=} \PY{n}{df}\PY{p}{[}\PY{n}{df}\PY{p}{[}\PY{l+s+s1}{\PYZsq{}}\PY{l+s+s1}{pclass}\PY{l+s+s1}{\PYZsq{}}\PY{p}{]} \PY{o}{==} \PY{l+m+mi}{3}\PY{p}{]}\PY{p}{[}\PY{l+s+s1}{\PYZsq{}}\PY{l+s+s1}{fare}\PY{l+s+s1}{\PYZsq{}}\PY{p}{]}

\PY{c+c1}{\PYZsh{} Tworzenie histogramu}
\PY{n}{plt}\PY{o}{.}\PY{n}{figure}\PY{p}{(}\PY{n}{figsize}\PY{o}{=}\PY{p}{(}\PY{l+m+mi}{10}\PY{p}{,} \PY{l+m+mi}{3}\PY{p}{)}\PY{p}{)}
\PY{n}{plt}\PY{o}{.}\PY{n}{hist}\PY{p}{(}\PY{n}{class\PYZus{}1}\PY{p}{,} \PY{n}{bins}\PY{o}{=}\PY{l+m+mi}{30}\PY{p}{,} \PY{n}{alpha}\PY{o}{=}\PY{l+m+mf}{0.7}\PY{p}{,} \PY{n}{color}\PY{o}{=}\PY{l+s+s1}{\PYZsq{}}\PY{l+s+s1}{gold}\PY{l+s+s1}{\PYZsq{}}\PY{p}{,} \PY{n}{label}\PY{o}{=}\PY{l+s+s1}{\PYZsq{}}\PY{l+s+s1}{1 klasa}\PY{l+s+s1}{\PYZsq{}}\PY{p}{,} \PY{n}{edgecolor}\PY{o}{=}\PY{l+s+s1}{\PYZsq{}}\PY{l+s+s1}{black}\PY{l+s+s1}{\PYZsq{}}\PY{p}{)}
\PY{n}{plt}\PY{o}{.}\PY{n}{hist}\PY{p}{(}\PY{n}{class\PYZus{}2}\PY{p}{,} \PY{n}{bins}\PY{o}{=}\PY{l+m+mi}{30}\PY{p}{,} \PY{n}{alpha}\PY{o}{=}\PY{l+m+mf}{0.7}\PY{p}{,} \PY{n}{color}\PY{o}{=}\PY{l+s+s1}{\PYZsq{}}\PY{l+s+s1}{silver}\PY{l+s+s1}{\PYZsq{}}\PY{p}{,} \PY{n}{label}\PY{o}{=}\PY{l+s+s1}{\PYZsq{}}\PY{l+s+s1}{2 klasa}\PY{l+s+s1}{\PYZsq{}}\PY{p}{,} \PY{n}{edgecolor}\PY{o}{=}\PY{l+s+s1}{\PYZsq{}}\PY{l+s+s1}{black}\PY{l+s+s1}{\PYZsq{}}\PY{p}{)}
\PY{n}{plt}\PY{o}{.}\PY{n}{hist}\PY{p}{(}\PY{n}{class\PYZus{}3}\PY{p}{,} \PY{n}{bins}\PY{o}{=}\PY{l+m+mi}{30}\PY{p}{,} \PY{n}{alpha}\PY{o}{=}\PY{l+m+mf}{0.7}\PY{p}{,} \PY{n}{color}\PY{o}{=}\PY{l+s+s1}{\PYZsq{}}\PY{l+s+s1}{brown}\PY{l+s+s1}{\PYZsq{}}\PY{p}{,} \PY{n}{label}\PY{o}{=}\PY{l+s+s1}{\PYZsq{}}\PY{l+s+s1}{3 klasa}\PY{l+s+s1}{\PYZsq{}}\PY{p}{,} \PY{n}{edgecolor}\PY{o}{=}\PY{l+s+s1}{\PYZsq{}}\PY{l+s+s1}{black}\PY{l+s+s1}{\PYZsq{}}\PY{p}{)}

\PY{c+c1}{\PYZsh{} Ustawienia wykresu}
\PY{n}{plt}\PY{o}{.}\PY{n}{title}\PY{p}{(}\PY{l+s+s1}{\PYZsq{}}\PY{l+s+s1}{Rozkład cen biletów w zależności od klasy biletu}\PY{l+s+s1}{\PYZsq{}}\PY{p}{)}
\PY{n}{plt}\PY{o}{.}\PY{n}{xlabel}\PY{p}{(}\PY{l+s+s1}{\PYZsq{}}\PY{l+s+s1}{Cena biletu}\PY{l+s+s1}{\PYZsq{}}\PY{p}{)}
\PY{n}{plt}\PY{o}{.}\PY{n}{xlim}\PY{p}{(}\PY{l+m+mi}{0}\PY{p}{,}\PY{l+m+mi}{300}\PY{p}{)}
\PY{n}{plt}\PY{o}{.}\PY{n}{ylabel}\PY{p}{(}\PY{l+s+s1}{\PYZsq{}}\PY{l+s+s1}{Liczba pasażerów}\PY{l+s+s1}{\PYZsq{}}\PY{p}{)}
\PY{n}{plt}\PY{o}{.}\PY{n}{ylim}\PY{p}{(}\PY{l+m+mi}{0}\PY{p}{,}\PY{l+m+mi}{100}\PY{p}{)}
\PY{n}{plt}\PY{o}{.}\PY{n}{legend}\PY{p}{(}\PY{p}{)}
\PY{n}{plt}\PY{o}{.}\PY{n}{grid}\PY{p}{(}\PY{n}{axis}\PY{o}{=}\PY{l+s+s1}{\PYZsq{}}\PY{l+s+s1}{y}\PY{l+s+s1}{\PYZsq{}}\PY{p}{,} \PY{n}{linestyle}\PY{o}{=}\PY{l+s+s1}{\PYZsq{}}\PY{l+s+s1}{\PYZhy{}\PYZhy{}}\PY{l+s+s1}{\PYZsq{}}\PY{p}{,} \PY{n}{alpha}\PY{o}{=}\PY{l+m+mf}{0.7}\PY{p}{)}

\PY{c+c1}{\PYZsh{} Wyświetlenie wykresu}
\PY{n}{plt}\PY{o}{.}\PY{n}{show}\PY{p}{(}\PY{p}{)}
\end{Verbatim}
\end{tcolorbox}

    \begin{center}
    \adjustimage{max size={0.9\linewidth}{0.9\paperheight}}{titanic_prezentacja_files/titanic_prezentacja_65_0.png}
    \end{center}
    { \hspace*{\fill} \\}
    
    \paragraph{CABIN - numer kabiny}\label{cabin---numer-kabiny}

    \begin{tcolorbox}[breakable, size=fbox, boxrule=1pt, pad at break*=1mm,colback=cellbackground, colframe=cellborder]
\prompt{In}{incolor}{95}{\boxspacing}
\begin{Verbatim}[commandchars=\\\{\}]
\PY{n}{pd}\PY{o}{.}\PY{n}{DataFrame}\PY{p}{(}\PY{n}{df}\PY{p}{[}\PY{l+s+s1}{\PYZsq{}}\PY{l+s+s1}{cabin}\PY{l+s+s1}{\PYZsq{}}\PY{p}{]}\PY{o}{.}\PY{n}{describe}\PY{p}{(}\PY{p}{)}\PY{p}{)}
\end{Verbatim}
\end{tcolorbox}

            \begin{tcolorbox}[breakable, size=fbox, boxrule=.5pt, pad at break*=1mm, opacityfill=0]
\prompt{Out}{outcolor}{95}{\boxspacing}
\begin{Verbatim}[commandchars=\\\{\}]
              cabin
count           295
unique          186
top     C23 C25 C27
freq              6
\end{Verbatim}
\end{tcolorbox}
        
    \subparagraph{Mamy informacje o 295 kabinach, które posiadają 186
wartości
unikatowych.}\label{mamy-informacje-o-295-kabinach-ktuxf3re-posiadajux105-186-wartoux15bci-unikatowych.}

    \paragraph{EMBARKED - port wejścia na
pokład}\label{embarked---port-wejux15bcia-na-pokux142ad}

    \begin{tcolorbox}[breakable, size=fbox, boxrule=1pt, pad at break*=1mm,colback=cellbackground, colframe=cellborder]
\prompt{In}{incolor}{96}{\boxspacing}
\begin{Verbatim}[commandchars=\\\{\}]
\PY{n}{pd}\PY{o}{.}\PY{n}{DataFrame}\PY{p}{(}\PY{n}{df}\PY{p}{[}\PY{l+s+s1}{\PYZsq{}}\PY{l+s+s1}{embarked}\PY{l+s+s1}{\PYZsq{}}\PY{p}{]}\PY{o}{.}\PY{n}{value\PYZus{}counts}\PY{p}{(}\PY{p}{)}\PY{p}{)}
\end{Verbatim}
\end{tcolorbox}

            \begin{tcolorbox}[breakable, size=fbox, boxrule=.5pt, pad at break*=1mm, opacityfill=0]
\prompt{Out}{outcolor}{96}{\boxspacing}
\begin{Verbatim}[commandchars=\\\{\}]
          count
embarked
S           914
C           270
Q           123
\end{Verbatim}
\end{tcolorbox}
        
    \subparagraph{Mamy dane na temat 3 portów, w których pasażerowie
wchodzili na
pokład.}\label{mamy-dane-na-temat-3-portuxf3w-w-ktuxf3rych-pasaux17cerowie-wchodzili-na-pokux142ad.}

\subparagraph{S = Southampton - 914
pasażerów}\label{s-southampton---914-pasaux17ceruxf3w}

\subparagraph{C = Cherbourg - 270
paseżerów}\label{c-cherbourg---270-paseux17ceruxf3w}

\subparagraph{Q = Queenstown - 123
pasażerów}\label{q-queenstown---123-pasaux17ceruxf3w}

    \paragraph{BOAT - numer łodzi
ratunkowej}\label{boat---numer-ux142odzi-ratunkowej}

    \begin{tcolorbox}[breakable, size=fbox, boxrule=1pt, pad at break*=1mm,colback=cellbackground, colframe=cellborder]
\prompt{In}{incolor}{97}{\boxspacing}
\begin{Verbatim}[commandchars=\\\{\}]
\PY{n}{pd}\PY{o}{.}\PY{n}{DataFrame}\PY{p}{(}\PY{n}{df}\PY{p}{[}\PY{l+s+s1}{\PYZsq{}}\PY{l+s+s1}{boat}\PY{l+s+s1}{\PYZsq{}}\PY{p}{]}\PY{o}{.}\PY{n}{describe}\PY{p}{(}\PY{p}{)}\PY{p}{)}
\end{Verbatim}
\end{tcolorbox}

            \begin{tcolorbox}[breakable, size=fbox, boxrule=.5pt, pad at break*=1mm, opacityfill=0]
\prompt{Out}{outcolor}{97}{\boxspacing}
\begin{Verbatim}[commandchars=\\\{\}]
       boat
count   486
unique   27
top      13
freq     39
\end{Verbatim}
\end{tcolorbox}
        
    \subparagraph{Mamy informacje o 27 unikatowych numerach łodzi
ratunkowcyh.}\label{mamy-informacje-o-27-unikatowych-numerach-ux142odzi-ratunkowcyh.}

    \paragraph{BODY - numer ciała jeśli pasażer nie przeżył i ciało zostało
odnalezione}\label{body---numer-ciaux142a-jeux15bli-pasaux17cer-nie-przeux17cyux142-i-ciaux142o-zostaux142o-odnalezione}

    \begin{tcolorbox}[breakable, size=fbox, boxrule=1pt, pad at break*=1mm,colback=cellbackground, colframe=cellborder]
\prompt{In}{incolor}{98}{\boxspacing}
\begin{Verbatim}[commandchars=\\\{\}]
\PY{n}{pd}\PY{o}{.}\PY{n}{DataFrame}\PY{p}{(}\PY{n}{df}\PY{p}{[}\PY{l+s+s1}{\PYZsq{}}\PY{l+s+s1}{body}\PY{l+s+s1}{\PYZsq{}}\PY{p}{]}\PY{o}{.}\PY{n}{unique}\PY{p}{(}\PY{p}{)}\PY{p}{)}
\end{Verbatim}
\end{tcolorbox}

            \begin{tcolorbox}[breakable, size=fbox, boxrule=.5pt, pad at break*=1mm, opacityfill=0]
\prompt{Out}{outcolor}{98}{\boxspacing}
\begin{Verbatim}[commandchars=\\\{\}]
         0
0      NaN
1    135.0
2     22.0
3    124.0
4    148.0
..     {\ldots}
117   14.0
118  131.0
119  312.0
120  328.0
121  304.0

[122 rows x 1 columns]
\end{Verbatim}
\end{tcolorbox}
        
    \subparagraph{Mamy informacje o 121 unikatowych numerach odnalezionych
ciał.}\label{mamy-informacje-o-121-unikatowych-numerach-odnalezionych-ciaux142.}

    \paragraph{HOME.DEST - miejsce docelowe
podróżujących}\label{home.dest---miejsce-docelowe-podruxf3ux17cujux105cych}

    \begin{tcolorbox}[breakable, size=fbox, boxrule=1pt, pad at break*=1mm,colback=cellbackground, colframe=cellborder]
\prompt{In}{incolor}{99}{\boxspacing}
\begin{Verbatim}[commandchars=\\\{\}]
\PY{n}{show}\PY{p}{(}\PY{n}{df}\PY{p}{[}\PY{l+s+s1}{\PYZsq{}}\PY{l+s+s1}{home.dest}\PY{l+s+s1}{\PYZsq{}}\PY{p}{]}\PY{o}{.}\PY{n}{unique}\PY{p}{(}\PY{p}{)}\PY{p}{)}
\end{Verbatim}
\end{tcolorbox}

    
    \begin{Verbatim}[commandchars=\\\{\}]
<IPython.core.display.HTML object>
    \end{Verbatim}

    
    \paragraph{HOME.DEST - miejsce docelowe
podróżujących}\label{home.dest---miejsce-docelowe-podruxf3ux17cujux105cych}

    \subparagraph{Mamy informacje o 370 unikatowych miejsach docelowych.
Jednak po przeglądzie częśći rekordów, widać, że niektóre częściowo
powtarzają się, poprzez podanie np. 2 miejsc docelowych (London/NY,
itd.)}\label{mamy-informacje-o-370-unikatowych-miejsach-docelowych.-jednak-po-przeglux105dzie-czux119ux15bux107i-rekorduxf3w-widaux107-ux17ce-niektuxf3re-czux119ux15bciowo-powtarzajux105-siux119-poprzez-podanie-np.-2-miejsc-docelowych-londonny-itd.}

    \section{4. Transformacja danych.}\label{transformacja-danych.}

    \paragraph{Duplikaty}\label{duplikaty}

    \begin{tcolorbox}[breakable, size=fbox, boxrule=1pt, pad at break*=1mm,colback=cellbackground, colframe=cellborder]
\prompt{In}{incolor}{100}{\boxspacing}
\begin{Verbatim}[commandchars=\\\{\}]
\PY{n}{pd}\PY{o}{.}\PY{n}{DataFrame}\PY{p}{(}\PY{n}{df}\PY{p}{[}\PY{n}{df}\PY{o}{.}\PY{n}{duplicated}\PY{p}{(}\PY{n}{subset}\PY{o}{=}\PY{l+s+s1}{\PYZsq{}}\PY{l+s+s1}{ticket}\PY{l+s+s1}{\PYZsq{}}\PY{p}{,} \PY{n}{keep}\PY{o}{=}\PY{k+kc}{False}\PY{p}{)}\PY{p}{]}\PY{p}{)}
\end{Verbatim}
\end{tcolorbox}

            \begin{tcolorbox}[breakable, size=fbox, boxrule=.5pt, pad at break*=1mm, opacityfill=0]
\prompt{Out}{outcolor}{100}{\boxspacing}
\begin{Verbatim}[commandchars=\\\{\}]
      pclass  survived                                             name  \textbackslash{}
0        1.0       1.0                    Allen, Miss. Elisabeth Walton
1        1.0       1.0                   Allison, Master. Hudson Trevor
2        1.0       0.0                     Allison, Miss. Helen Loraine
3        1.0       0.0             Allison, Mr. Hudson Joshua Creighton
4        1.0       0.0  Allison, Mrs. Hudson J C (Bessie Waldo Daniels)
{\ldots}      {\ldots}       {\ldots}                                              {\ldots}
1299     3.0       0.0                              Yasbeck, Mr. Antoni
1300     3.0       1.0          Yasbeck, Mrs. Antoni (Selini Alexander)
1303     3.0       0.0                            Yousseff, Mr. Gerious
1304     3.0       0.0                             Zabour, Miss. Hileni
1305     3.0       0.0                            Zabour, Miss. Thamine

         sex  age\_raw  sibsp  parch  ticket      fare    cabin embarked boat  \textbackslash{}
0     female  29.0000    0.0    0.0   24160  211.3375       B5        S    2
1       male   0.9167    1.0    2.0  113781  151.5500  C22 C26        S   11
2     female   2.0000    1.0    2.0  113781  151.5500  C22 C26        S  NaN
3       male  30.0000    1.0    2.0  113781  151.5500  C22 C26        S  NaN
4     female  25.0000    1.0    2.0  113781  151.5500  C22 C26        S  NaN
{\ldots}      {\ldots}      {\ldots}    {\ldots}    {\ldots}     {\ldots}       {\ldots}      {\ldots}      {\ldots}  {\ldots}
1299    male  27.0000    1.0    0.0    2659   14.4542      NaN        C    C
1300  female  15.0000    1.0    0.0    2659   14.4542      NaN        C  NaN
1303    male      NaN    0.0    0.0    2627   14.4583      NaN        C  NaN
1304  female  14.5000    1.0    0.0    2665   14.4542      NaN        C  NaN
1305  female      NaN    1.0    0.0    2665   14.4542      NaN        C  NaN

       body                        home.dest   age
0       NaN                     St Louis, MO  29.0
1       NaN  Montreal, PQ / Chesterville, ON   1.0
2       NaN  Montreal, PQ / Chesterville, ON   2.0
3     135.0  Montreal, PQ / Chesterville, ON  30.0
4       NaN  Montreal, PQ / Chesterville, ON  25.0
{\ldots}     {\ldots}                              {\ldots}   {\ldots}
1299    NaN                              NaN  27.0
1300    NaN                              NaN  15.0
1303    NaN                              NaN   NaN
1304  328.0                              NaN  14.0
1305    NaN                              NaN   NaN

[596 rows x 15 columns]
\end{Verbatim}
\end{tcolorbox}
        
    \subparagraph{W numeracji biletów występują identyczne numery, jednak są
przypisane do różnych osób o podobnych nazwiskach, co pozwala sądzić, że
na jeden bilet przypisany był do kilku osób, np.
rodziny.}\label{w-numeracji-biletuxf3w-wystux119pujux105-identyczne-numery-jednak-sux105-przypisane-do-ruxf3ux17cnych-osuxf3b-o-podobnych-nazwiskach-co-pozwala-sux105dziux107-ux17ce-na-jeden-bilet-przypisany-byux142-do-kilku-osuxf3b-np.-rodziny.}

    \begin{tcolorbox}[breakable, size=fbox, boxrule=1pt, pad at break*=1mm,colback=cellbackground, colframe=cellborder]
\prompt{In}{incolor}{101}{\boxspacing}
\begin{Verbatim}[commandchars=\\\{\}]
\PY{n}{pd}\PY{o}{.}\PY{n}{DataFrame}\PY{p}{(}\PY{n}{df}\PY{p}{[}\PY{n}{df}\PY{o}{.}\PY{n}{duplicated}\PY{p}{(}\PY{n}{subset}\PY{o}{=}\PY{l+s+s1}{\PYZsq{}}\PY{l+s+s1}{name}\PY{l+s+s1}{\PYZsq{}}\PY{p}{,} \PY{n}{keep}\PY{o}{=}\PY{k+kc}{False}\PY{p}{)}\PY{p}{]}\PY{p}{)}
\end{Verbatim}
\end{tcolorbox}

            \begin{tcolorbox}[breakable, size=fbox, boxrule=.5pt, pad at break*=1mm, opacityfill=0]
\prompt{Out}{outcolor}{101}{\boxspacing}
\begin{Verbatim}[commandchars=\\\{\}]
     pclass  survived                  name     sex  age\_raw  sibsp  parch  \textbackslash{}
725     3.0       1.0  Connolly, Miss. Kate  female     22.0    0.0    0.0
726     3.0       0.0  Connolly, Miss. Kate  female     30.0    0.0    0.0
924     3.0       0.0      Kelly, Mr. James    male     34.5    0.0    0.0
925     3.0       0.0      Kelly, Mr. James    male     44.0    0.0    0.0

     ticket    fare cabin embarked boat  body home.dest   age
725  370373  7.7500   NaN        Q   13   NaN   Ireland  22.0
726  330972  7.6292   NaN        Q  NaN   NaN   Ireland  30.0
924  330911  7.8292   NaN        Q  NaN  70.0       NaN  34.0
925  363592  8.0500   NaN        S  NaN   NaN       NaN  44.0
\end{Verbatim}
\end{tcolorbox}
        
    \subparagraph{Występują dwa identyczne nazwiska, jednak posiadają różne
dane odnośnie wieku i numeru biletu. Można zatem stwierdzić, że nie są
duplikatami.}\label{wystux119pujux105-dwa-identyczne-nazwiska-jednak-posiadajux105-ruxf3ux17cne-dane-odnoux15bnie-wieku-i-numeru-biletu.-moux17cna-zatem-stwierdziux107-ux17ce-nie-sux105-duplikatami.}

    \paragraph{Naprawa brakujących
wartości}\label{naprawa-brakujux105cych-wartoux15bci}

    \subparagraph{AGE - Wypełnienie brakujących wartości wieku, średnią
arytmetyczną dla kobiet i
mężczyzn}\label{age---wypeux142nienie-brakujux105cych-wartoux15bci-wieku-ux15bredniux105-arytmetycznux105-dla-kobiet-i-mux119ux17cczyzn}

    \begin{tcolorbox}[breakable, size=fbox, boxrule=1pt, pad at break*=1mm,colback=cellbackground, colframe=cellborder]
\prompt{In}{incolor}{102}{\boxspacing}
\begin{Verbatim}[commandchars=\\\{\}]
\PY{c+c1}{\PYZsh{} Obliczenie średniego wieku dla każdej płci}
\PY{n}{age\PYZus{}mean\PYZus{}f} \PY{o}{=} \PY{n}{df}\PY{p}{[}\PY{n}{df}\PY{p}{[}\PY{l+s+s1}{\PYZsq{}}\PY{l+s+s1}{sex}\PY{l+s+s1}{\PYZsq{}}\PY{p}{]} \PY{o}{==} \PY{l+s+s1}{\PYZsq{}}\PY{l+s+s1}{female}\PY{l+s+s1}{\PYZsq{}}\PY{p}{]}\PY{p}{[}\PY{l+s+s1}{\PYZsq{}}\PY{l+s+s1}{age}\PY{l+s+s1}{\PYZsq{}}\PY{p}{]}\PY{o}{.}\PY{n}{mean}\PY{p}{(}\PY{p}{)}\PY{o}{.}\PY{n}{round}\PY{p}{(}\PY{p}{)}
\PY{n}{age\PYZus{}mean\PYZus{}m} \PY{o}{=} \PY{n}{df}\PY{p}{[}\PY{n}{df}\PY{p}{[}\PY{l+s+s1}{\PYZsq{}}\PY{l+s+s1}{sex}\PY{l+s+s1}{\PYZsq{}}\PY{p}{]} \PY{o}{==} \PY{l+s+s1}{\PYZsq{}}\PY{l+s+s1}{male}\PY{l+s+s1}{\PYZsq{}}\PY{p}{]}\PY{p}{[}\PY{l+s+s1}{\PYZsq{}}\PY{l+s+s1}{age}\PY{l+s+s1}{\PYZsq{}}\PY{p}{]}\PY{o}{.}\PY{n}{mean}\PY{p}{(}\PY{p}{)}\PY{o}{.}\PY{n}{round}\PY{p}{(}\PY{p}{)}

\PY{c+c1}{\PYZsh{} Uzupełnienie brakujących wartości}
\PY{n}{df}\PY{o}{.}\PY{n}{loc}\PY{p}{[}\PY{n}{df}\PY{p}{[}\PY{l+s+s1}{\PYZsq{}}\PY{l+s+s1}{sex}\PY{l+s+s1}{\PYZsq{}}\PY{p}{]} \PY{o}{==} \PY{l+s+s1}{\PYZsq{}}\PY{l+s+s1}{female}\PY{l+s+s1}{\PYZsq{}}\PY{p}{,} \PY{l+s+s1}{\PYZsq{}}\PY{l+s+s1}{age}\PY{l+s+s1}{\PYZsq{}}\PY{p}{]} \PY{o}{=} \PY{n}{df}\PY{o}{.}\PY{n}{loc}\PY{p}{[}\PY{n}{df}\PY{p}{[}\PY{l+s+s1}{\PYZsq{}}\PY{l+s+s1}{sex}\PY{l+s+s1}{\PYZsq{}}\PY{p}{]} \PY{o}{==} \PY{l+s+s1}{\PYZsq{}}\PY{l+s+s1}{female}\PY{l+s+s1}{\PYZsq{}}\PY{p}{,} \PY{l+s+s1}{\PYZsq{}}\PY{l+s+s1}{age}\PY{l+s+s1}{\PYZsq{}}\PY{p}{]}\PY{o}{.}\PY{n}{fillna}\PY{p}{(}\PY{n}{age\PYZus{}mean\PYZus{}f}\PY{p}{)}
\PY{n}{df}\PY{o}{.}\PY{n}{loc}\PY{p}{[}\PY{n}{df}\PY{p}{[}\PY{l+s+s1}{\PYZsq{}}\PY{l+s+s1}{sex}\PY{l+s+s1}{\PYZsq{}}\PY{p}{]} \PY{o}{==} \PY{l+s+s1}{\PYZsq{}}\PY{l+s+s1}{male}\PY{l+s+s1}{\PYZsq{}}\PY{p}{,} \PY{l+s+s1}{\PYZsq{}}\PY{l+s+s1}{age}\PY{l+s+s1}{\PYZsq{}}\PY{p}{]} \PY{o}{=} \PY{n}{df}\PY{o}{.}\PY{n}{loc}\PY{p}{[}\PY{n}{df}\PY{p}{[}\PY{l+s+s1}{\PYZsq{}}\PY{l+s+s1}{sex}\PY{l+s+s1}{\PYZsq{}}\PY{p}{]} \PY{o}{==} \PY{l+s+s1}{\PYZsq{}}\PY{l+s+s1}{male}\PY{l+s+s1}{\PYZsq{}}\PY{p}{,} \PY{l+s+s1}{\PYZsq{}}\PY{l+s+s1}{age}\PY{l+s+s1}{\PYZsq{}}\PY{p}{]}\PY{o}{.}\PY{n}{fillna}\PY{p}{(}\PY{n}{age\PYZus{}mean\PYZus{}m}\PY{p}{)}

\PY{c+c1}{\PYZsh{} Sprawdzenie, czy nadal są puste wartości}
\PY{n}{missing\PYZus{}age} \PY{o}{=} \PY{n}{df}\PY{p}{[}\PY{n}{df}\PY{p}{[}\PY{l+s+s1}{\PYZsq{}}\PY{l+s+s1}{age}\PY{l+s+s1}{\PYZsq{}}\PY{p}{]}\PY{o}{.}\PY{n}{isna}\PY{p}{(}\PY{p}{)}\PY{p}{]}\PY{p}{[}\PY{l+s+s1}{\PYZsq{}}\PY{l+s+s1}{sex}\PY{l+s+s1}{\PYZsq{}}\PY{p}{]}\PY{o}{.}\PY{n}{value\PYZus{}counts}\PY{p}{(}\PY{p}{)}
\PY{n}{pd}\PY{o}{.}\PY{n}{DataFrame}\PY{p}{(}\PY{n}{missing\PYZus{}age}\PY{p}{)}
\end{Verbatim}
\end{tcolorbox}

            \begin{tcolorbox}[breakable, size=fbox, boxrule=.5pt, pad at break*=1mm, opacityfill=0]
\prompt{Out}{outcolor}{102}{\boxspacing}
\begin{Verbatim}[commandchars=\\\{\}]
Empty DataFrame
Columns: [count]
Index: []
\end{Verbatim}
\end{tcolorbox}
        
    \begin{tcolorbox}[breakable, size=fbox, boxrule=1pt, pad at break*=1mm,colback=cellbackground, colframe=cellborder]
\prompt{In}{incolor}{103}{\boxspacing}
\begin{Verbatim}[commandchars=\\\{\}]
\PY{n+nb}{print}\PY{p}{(}\PY{l+s+sa}{f}\PY{l+s+s2}{\PYZdq{}}\PY{l+s+s2}{Średnia arytmetyczna dla kobiet }\PY{l+s+si}{\PYZob{}}\PY{n}{age\PYZus{}mean\PYZus{}f}\PY{l+s+si}{\PYZcb{}}\PY{l+s+s2}{\PYZdq{}}\PY{p}{)}
\end{Verbatim}
\end{tcolorbox}

    \begin{Verbatim}[commandchars=\\\{\}]
Średnia arytmetyczna dla kobiet 29.0
    \end{Verbatim}

    \begin{tcolorbox}[breakable, size=fbox, boxrule=1pt, pad at break*=1mm,colback=cellbackground, colframe=cellborder]
\prompt{In}{incolor}{104}{\boxspacing}
\begin{Verbatim}[commandchars=\\\{\}]
\PY{n+nb}{print}\PY{p}{(}\PY{l+s+sa}{f}\PY{l+s+s2}{\PYZdq{}}\PY{l+s+s2}{Średnia arytmetyczna dla mężczyzn }\PY{l+s+si}{\PYZob{}}\PY{n}{age\PYZus{}mean\PYZus{}m}\PY{l+s+si}{\PYZcb{}}\PY{l+s+s2}{\PYZdq{}}\PY{p}{)}
\end{Verbatim}
\end{tcolorbox}

    \begin{Verbatim}[commandchars=\\\{\}]
Średnia arytmetyczna dla mężczyzn 31.0
    \end{Verbatim}

    \subparagraph{FARE - Wypełnienie brakujących wartości ceny biletu,
średnią
arytmetyczną.}\label{fare---wypeux142nienie-brakujux105cych-wartoux15bci-ceny-biletu-ux15bredniux105-arytmetycznux105.}

    \begin{tcolorbox}[breakable, size=fbox, boxrule=1pt, pad at break*=1mm,colback=cellbackground, colframe=cellborder]
\prompt{In}{incolor}{105}{\boxspacing}
\begin{Verbatim}[commandchars=\\\{\}]
\PY{n}{fare\PYZus{}na} \PY{o}{=} \PY{n}{df}\PY{p}{[}\PY{n}{df}\PY{p}{[}\PY{l+s+s1}{\PYZsq{}}\PY{l+s+s1}{fare}\PY{l+s+s1}{\PYZsq{}}\PY{p}{]}\PY{o}{.}\PY{n}{isna}\PY{p}{(}\PY{p}{)}\PY{p}{]}
\PY{n+nb}{print}\PY{p}{(}\PY{l+s+sa}{f}\PY{l+s+s2}{\PYZdq{}}\PY{l+s+s2}{Brakujące wartości mamy dla 3 klasy}\PY{l+s+s2}{\PYZdq{}}\PY{p}{)}
\end{Verbatim}
\end{tcolorbox}

    \begin{Verbatim}[commandchars=\\\{\}]
Brakujące wartości mamy dla 3 klasy
    \end{Verbatim}

    \begin{tcolorbox}[breakable, size=fbox, boxrule=1pt, pad at break*=1mm,colback=cellbackground, colframe=cellborder]
\prompt{In}{incolor}{106}{\boxspacing}
\begin{Verbatim}[commandchars=\\\{\}]
\PY{n}{fare\PYZus{}mean} \PY{o}{=} \PY{n}{df}\PY{p}{[}\PY{n}{df}\PY{p}{[}\PY{l+s+s1}{\PYZsq{}}\PY{l+s+s1}{pclass}\PY{l+s+s1}{\PYZsq{}}\PY{p}{]} \PY{o}{==} \PY{l+m+mi}{3}\PY{p}{]}\PY{p}{[}\PY{l+s+s1}{\PYZsq{}}\PY{l+s+s1}{fare}\PY{l+s+s1}{\PYZsq{}}\PY{p}{]}\PY{o}{.}\PY{n}{mean}\PY{p}{(}\PY{p}{)}\PY{o}{.}\PY{n}{round}\PY{p}{(}\PY{l+m+mi}{2}\PY{p}{)}
\PY{n+nb}{print}\PY{p}{(}\PY{l+s+sa}{f}\PY{l+s+s2}{\PYZdq{}}\PY{l+s+s2}{Średnia arytmetyczna w 3 klasie }\PY{l+s+si}{\PYZob{}}\PY{n}{fare\PYZus{}mean}\PY{l+s+si}{\PYZcb{}}\PY{l+s+s2}{\PYZdq{}}\PY{p}{)}
\end{Verbatim}
\end{tcolorbox}

    \begin{Verbatim}[commandchars=\\\{\}]
Średnia arytmetyczna w 3 klasie 13.3
    \end{Verbatim}

    \begin{tcolorbox}[breakable, size=fbox, boxrule=1pt, pad at break*=1mm,colback=cellbackground, colframe=cellborder]
\prompt{In}{incolor}{107}{\boxspacing}
\begin{Verbatim}[commandchars=\\\{\}]
\PY{c+c1}{\PYZsh{} Obliczenie średniej ceny biletu w 3 klasie}
\PY{n}{fare\PYZus{}mean} \PY{o}{=} \PY{n}{df}\PY{p}{[}\PY{n}{df}\PY{p}{[}\PY{l+s+s1}{\PYZsq{}}\PY{l+s+s1}{pclass}\PY{l+s+s1}{\PYZsq{}}\PY{p}{]} \PY{o}{==} \PY{l+m+mi}{3}\PY{p}{]}\PY{p}{[}\PY{l+s+s1}{\PYZsq{}}\PY{l+s+s1}{fare}\PY{l+s+s1}{\PYZsq{}}\PY{p}{]}\PY{o}{.}\PY{n}{mean}\PY{p}{(}\PY{p}{)}\PY{o}{.}\PY{n}{round}\PY{p}{(}\PY{l+m+mi}{2}\PY{p}{)}

\PY{c+c1}{\PYZsh{} Uzupełnienie brakujących wartości}
\PY{n}{df}\PY{p}{[}\PY{l+s+s1}{\PYZsq{}}\PY{l+s+s1}{fare}\PY{l+s+s1}{\PYZsq{}}\PY{p}{]} \PY{o}{=} \PY{n}{df}\PY{p}{[}\PY{l+s+s1}{\PYZsq{}}\PY{l+s+s1}{fare}\PY{l+s+s1}{\PYZsq{}}\PY{p}{]}\PY{o}{.}\PY{n}{fillna}\PY{p}{(}\PY{n}{fare\PYZus{}mean}\PY{p}{)}


\PY{c+c1}{\PYZsh{} Sprawdzenie, czy nadal są puste wartości}
\PY{n}{fare\PYZus{}na} \PY{o}{=} \PY{n}{df}\PY{p}{[}\PY{n}{df}\PY{p}{[}\PY{l+s+s1}{\PYZsq{}}\PY{l+s+s1}{fare}\PY{l+s+s1}{\PYZsq{}}\PY{p}{]}\PY{o}{.}\PY{n}{isna}\PY{p}{(}\PY{p}{)}\PY{p}{]}
\PY{n}{fare\PYZus{}na}
\end{Verbatim}
\end{tcolorbox}

            \begin{tcolorbox}[breakable, size=fbox, boxrule=.5pt, pad at break*=1mm, opacityfill=0]
\prompt{Out}{outcolor}{107}{\boxspacing}
\begin{Verbatim}[commandchars=\\\{\}]
Empty DataFrame
Columns: [pclass, survived, name, sex, age\_raw, sibsp, parch, ticket, fare,
cabin, embarked, boat, body, home.dest, age]
Index: []
\end{Verbatim}
\end{tcolorbox}
        
    \subparagraph{BOAT - sprawdzenie pustych wartości o łodzi ratunkowej dla
ocalałych
pasażerów}\label{boat---sprawdzenie-pustych-wartoux15bci-o-ux142odzi-ratunkowej-dla-ocalaux142ych-pasaux17ceruxf3w}

    \begin{tcolorbox}[breakable, size=fbox, boxrule=1pt, pad at break*=1mm,colback=cellbackground, colframe=cellborder]
\prompt{In}{incolor}{108}{\boxspacing}
\begin{Verbatim}[commandchars=\\\{\}]
\PY{n}{boat\PYZus{}na} \PY{o}{=} \PY{n}{df}\PY{p}{[}\PY{n}{df}\PY{p}{[}\PY{l+s+s1}{\PYZsq{}}\PY{l+s+s1}{survived}\PY{l+s+s1}{\PYZsq{}}\PY{p}{]} \PY{o}{==} \PY{l+m+mi}{1}\PY{p}{]}\PY{p}{[}\PY{l+s+s1}{\PYZsq{}}\PY{l+s+s1}{boat}\PY{l+s+s1}{\PYZsq{}}\PY{p}{]}\PY{o}{.}\PY{n}{isna}\PY{p}{(}\PY{p}{)}\PY{o}{.}\PY{n}{sum}\PY{p}{(}\PY{p}{)}
\PY{n}{pd}\PY{o}{.}\PY{n}{DataFrame}\PY{p}{(}\PY{p}{\PYZob{}}\PY{l+s+s1}{\PYZsq{}}\PY{l+s+s1}{Empty Boat Count}\PY{l+s+s1}{\PYZsq{}}\PY{p}{:} \PY{p}{[}\PY{n}{boat\PYZus{}na}\PY{p}{]}\PY{p}{\PYZcb{}}\PY{p}{)}
\end{Verbatim}
\end{tcolorbox}

            \begin{tcolorbox}[breakable, size=fbox, boxrule=.5pt, pad at break*=1mm, opacityfill=0]
\prompt{Out}{outcolor}{108}{\boxspacing}
\begin{Verbatim}[commandchars=\\\{\}]
   Empty Boat Count
0                23
\end{Verbatim}
\end{tcolorbox}
        
    \subparagraph{Występują puste wartości o nemarach łodzi ratunkowych, w
których byli ocaleni pasażerowie. Może to być wynikiem nieścisłości w
zbieraniu danych lub mogło być wynikiem uratowania pasażerów w inny
sposób.}\label{wystux119pujux105-puste-wartoux15bci-o-nemarach-ux142odzi-ratunkowych-w-ktuxf3rych-byli-ocaleni-pasaux17cerowie.-moux17ce-to-byux107-wynikiem-nieux15bcisux142oux15bci-w-zbieraniu-danych-lub-mogux142o-byux107-wynikiem-uratowania-pasaux17ceruxf3w-w-inny-sposuxf3b.}

    \section{5. Analiza relacji między
zmiennymi}\label{analiza-relacji-miux119dzy-zmiennymi}

    \paragraph{Klasa biletu, odsetek
ocalałych.}\label{klasa-biletu-odsetek-ocalaux142ych.}

    \begin{tcolorbox}[breakable, size=fbox, boxrule=1pt, pad at break*=1mm,colback=cellbackground, colframe=cellborder]
\prompt{In}{incolor}{109}{\boxspacing}
\begin{Verbatim}[commandchars=\\\{\}]
\PY{n}{plt}\PY{o}{.}\PY{n}{figure}\PY{p}{(}\PY{n}{figsize}\PY{o}{=}\PY{p}{(}\PY{l+m+mi}{8}\PY{p}{,} \PY{l+m+mi}{5}\PY{p}{)}\PY{p}{)}
\PY{n}{sns}\PY{o}{.}\PY{n}{barplot}\PY{p}{(}\PY{n}{x}\PY{o}{=}\PY{l+s+s1}{\PYZsq{}}\PY{l+s+s1}{pclass}\PY{l+s+s1}{\PYZsq{}}\PY{p}{,} \PY{n}{y}\PY{o}{=}\PY{l+s+s1}{\PYZsq{}}\PY{l+s+s1}{survived}\PY{l+s+s1}{\PYZsq{}}\PY{p}{,} \PY{n}{data}\PY{o}{=}\PY{n}{df}\PY{p}{,} \PY{n}{hue}\PY{o}{=}\PY{l+s+s1}{\PYZsq{}}\PY{l+s+s1}{pclass}\PY{l+s+s1}{\PYZsq{}}\PY{p}{,} \PY{n}{palette}\PY{o}{=}\PY{p}{\PYZob{}}\PY{l+m+mi}{1}\PY{p}{:} \PY{l+s+s1}{\PYZsq{}}\PY{l+s+s1}{\PYZsh{}FFD700}\PY{l+s+s1}{\PYZsq{}}\PY{p}{,} \PY{l+m+mi}{2}\PY{p}{:} \PY{l+s+s1}{\PYZsq{}}\PY{l+s+s1}{\PYZsh{}C0C0C0}\PY{l+s+s1}{\PYZsq{}}\PY{p}{,} \PY{l+m+mi}{3}\PY{p}{:} \PY{l+s+s1}{\PYZsq{}}\PY{l+s+s1}{\PYZsh{}8B4513}\PY{l+s+s1}{\PYZsq{}}\PY{p}{\PYZcb{}}\PY{p}{)}

\PY{n}{plt}\PY{o}{.}\PY{n}{title}\PY{p}{(}\PY{l+s+s1}{\PYZsq{}}\PY{l+s+s1}{Przeżywalność w zależności od klasy pasażerskiej}\PY{l+s+s1}{\PYZsq{}}\PY{p}{)}
\PY{n}{plt}\PY{o}{.}\PY{n}{xlabel}\PY{p}{(}\PY{l+s+s1}{\PYZsq{}}\PY{l+s+s1}{Klasa pasażerska}\PY{l+s+s1}{\PYZsq{}}\PY{p}{)}
\PY{n}{plt}\PY{o}{.}\PY{n}{ylabel}\PY{p}{(}\PY{l+s+s1}{\PYZsq{}}\PY{l+s+s1}{Odsetek przeżywalności}\PY{l+s+s1}{\PYZsq{}}\PY{p}{)}
\PY{n}{plt}\PY{o}{.}\PY{n}{ylim}\PY{p}{(}\PY{l+m+mi}{0}\PY{p}{,} \PY{l+m+mi}{1}\PY{p}{)}  \PY{c+c1}{\PYZsh{} Ograniczenie do wartości procentowych}
\PY{n}{plt}\PY{o}{.}\PY{n}{grid}\PY{p}{(}\PY{k+kc}{True}\PY{p}{,} \PY{n}{linestyle}\PY{o}{=}\PY{l+s+s1}{\PYZsq{}}\PY{l+s+s1}{\PYZhy{}\PYZhy{}}\PY{l+s+s1}{\PYZsq{}}\PY{p}{,} \PY{n}{alpha}\PY{o}{=}\PY{l+m+mf}{0.6}\PY{p}{)}
\PY{n}{plt}\PY{o}{.}\PY{n}{show}\PY{p}{(}\PY{p}{)}
\end{Verbatim}
\end{tcolorbox}

    \begin{center}
    \adjustimage{max size={0.9\linewidth}{0.9\paperheight}}{titanic_prezentacja_files/titanic_prezentacja_102_0.png}
    \end{center}
    { \hspace*{\fill} \\}
    
    \subparagraph{Wykres pokazuje, że pasażerowie podróżujący w wyższej
klasie, mieli większe szanse na przeżycie (1 - klasa najwyższa, 3 -
klasa
najniższa).}\label{wykres-pokazuje-ux17ce-pasaux17cerowie-podruxf3ux17cujux105cy-w-wyux17cszej-klasie-mieli-wiux119ksze-szanse-na-przeux17cycie-1---klasa-najwyux17csza-3---klasa-najniux17csza.}

    \paragraph{Cena biletu, klasa
pasażerska.}\label{cena-biletu-klasa-pasaux17cerska.}

    \begin{tcolorbox}[breakable, size=fbox, boxrule=1pt, pad at break*=1mm,colback=cellbackground, colframe=cellborder]
\prompt{In}{incolor}{110}{\boxspacing}
\begin{Verbatim}[commandchars=\\\{\}]
\PY{n}{df\PYZus{}temp} \PY{o}{=} \PY{n}{df}\PY{o}{.}\PY{n}{copy}\PY{p}{(}\PY{p}{)}
\PY{n}{df\PYZus{}temp}\PY{p}{[}\PY{l+s+s1}{\PYZsq{}}\PY{l+s+s1}{pclass}\PY{l+s+s1}{\PYZsq{}}\PY{p}{]} \PY{o}{=} \PY{n}{df\PYZus{}temp}\PY{p}{[}\PY{l+s+s1}{\PYZsq{}}\PY{l+s+s1}{pclass}\PY{l+s+s1}{\PYZsq{}}\PY{p}{]}\PY{o}{.}\PY{n}{astype}\PY{p}{(}\PY{n+nb}{str}\PY{p}{)}

\PY{n}{plt}\PY{o}{.}\PY{n}{figure}\PY{p}{(}\PY{n}{figsize}\PY{o}{=}\PY{p}{(}\PY{l+m+mi}{12}\PY{p}{,} \PY{l+m+mi}{4}\PY{p}{)}\PY{p}{)}
\PY{n}{sns}\PY{o}{.}\PY{n}{boxplot}\PY{p}{(}\PY{n}{x}\PY{o}{=}\PY{l+s+s1}{\PYZsq{}}\PY{l+s+s1}{fare}\PY{l+s+s1}{\PYZsq{}}\PY{p}{,} \PY{n}{y}\PY{o}{=}\PY{l+s+s1}{\PYZsq{}}\PY{l+s+s1}{pclass}\PY{l+s+s1}{\PYZsq{}}\PY{p}{,} \PY{n}{data}\PY{o}{=}\PY{n}{df\PYZus{}temp}\PY{p}{,} \PY{n}{hue}\PY{o}{=}\PY{l+s+s1}{\PYZsq{}}\PY{l+s+s1}{pclass}\PY{l+s+s1}{\PYZsq{}}\PY{p}{,} \PY{n}{vert}\PY{o}{=}\PY{k+kc}{False}\PY{p}{,}\PY{p}{)}

\PY{n}{plt}\PY{o}{.}\PY{n}{title}\PY{p}{(}\PY{l+s+s1}{\PYZsq{}}\PY{l+s+s1}{Cena biletu a klasa pasażerska}\PY{l+s+s1}{\PYZsq{}}\PY{p}{)}
\PY{n}{plt}\PY{o}{.}\PY{n}{xlabel}\PY{p}{(}\PY{l+s+s1}{\PYZsq{}}\PY{l+s+s1}{Cena biletu}\PY{l+s+s1}{\PYZsq{}}\PY{p}{)}
\PY{n}{plt}\PY{o}{.}\PY{n}{xlim}\PY{p}{(}\PY{o}{\PYZhy{}}\PY{l+m+mi}{10}\PY{p}{,} \PY{l+m+mi}{250}\PY{p}{)}
\PY{n}{plt}\PY{o}{.}\PY{n}{ylabel}\PY{p}{(}\PY{l+s+s1}{\PYZsq{}}\PY{l+s+s1}{Klasa pasażerska}\PY{l+s+s1}{\PYZsq{}}\PY{p}{)}
\PY{c+c1}{\PYZsh{}plt.xscale(\PYZsq{}log\PYZsq{})  \PYZsh{} Skala logarytmiczna dla czytelności}
\PY{n}{plt}\PY{o}{.}\PY{n}{grid}\PY{p}{(}\PY{k+kc}{True}\PY{p}{,} \PY{n}{linestyle}\PY{o}{=}\PY{l+s+s1}{\PYZsq{}}\PY{l+s+s1}{\PYZhy{}\PYZhy{}}\PY{l+s+s1}{\PYZsq{}}\PY{p}{,} \PY{n}{alpha}\PY{o}{=}\PY{l+m+mf}{0.6}\PY{p}{)}
\PY{n}{plt}\PY{o}{.}\PY{n}{show}\PY{p}{(}\PY{p}{)}
\end{Verbatim}
\end{tcolorbox}

    \begin{center}
    \adjustimage{max size={0.9\linewidth}{0.9\paperheight}}{titanic_prezentacja_files/titanic_prezentacja_105_0.png}
    \end{center}
    { \hspace*{\fill} \\}
    
    \subparagraph{Z wykresu wynika, że 75\% wawrtośći dla ceny biletu w
klasie 3, jest poniżej 50\% wartości cen biletu w klasie 2. Natomiast
większość wartości cen biletów z klasy 3 i ponad 75\% wartośći cen
biletów z klasy 2 jest poniżej 25\% wartości cen biletów w klasie
1.}\label{z-wykresu-wynika-ux17ce-75-wawrtoux15bux107i-dla-ceny-biletu-w-klasie-3-jest-poniux17cej-50-wartoux15bci-cen-biletu-w-klasie-2.-natomiast-wiux119kszoux15bux107-wartoux15bci-cen-biletuxf3w-z-klasy-3-i-ponad-75-wartoux15bux107i-cen-biletuxf3w-z-klasy-2-jest-poniux17cej-25-wartoux15bci-cen-biletuxf3w-w-klasie-1.}

    \paragraph{Cena biletu, ocalenie.}\label{cena-biletu-ocalenie.}

    \begin{tcolorbox}[breakable, size=fbox, boxrule=1pt, pad at break*=1mm,colback=cellbackground, colframe=cellborder]
\prompt{In}{incolor}{111}{\boxspacing}
\begin{Verbatim}[commandchars=\\\{\}]
\PY{n}{df\PYZus{}temp}\PY{p}{[}\PY{l+s+s1}{\PYZsq{}}\PY{l+s+s1}{survived}\PY{l+s+s1}{\PYZsq{}}\PY{p}{]} \PY{o}{=} \PY{n}{df\PYZus{}temp}\PY{p}{[}\PY{l+s+s1}{\PYZsq{}}\PY{l+s+s1}{survived}\PY{l+s+s1}{\PYZsq{}}\PY{p}{]}\PY{o}{.}\PY{n}{astype}\PY{p}{(}\PY{n+nb}{str}\PY{p}{)}

\PY{n}{plt}\PY{o}{.}\PY{n}{figure}\PY{p}{(}\PY{n}{figsize}\PY{o}{=}\PY{p}{(}\PY{l+m+mi}{12}\PY{p}{,} \PY{l+m+mi}{5}\PY{p}{)}\PY{p}{)}
\PY{n}{sns}\PY{o}{.}\PY{n}{boxplot}\PY{p}{(}\PY{n}{x}\PY{o}{=}\PY{l+s+s1}{\PYZsq{}}\PY{l+s+s1}{fare}\PY{l+s+s1}{\PYZsq{}}\PY{p}{,} \PY{n}{y}\PY{o}{=}\PY{l+s+s1}{\PYZsq{}}\PY{l+s+s1}{survived}\PY{l+s+s1}{\PYZsq{}}\PY{p}{,} \PY{n}{data}\PY{o}{=}\PY{n}{df\PYZus{}temp}\PY{p}{,} \PY{n}{hue}\PY{o}{=}\PY{l+s+s1}{\PYZsq{}}\PY{l+s+s1}{survived}\PY{l+s+s1}{\PYZsq{}}\PY{p}{,} \PY{n}{vert}\PY{o}{=}\PY{k+kc}{False}\PY{p}{)}

\PY{n}{plt}\PY{o}{.}\PY{n}{title}\PY{p}{(}\PY{l+s+s1}{\PYZsq{}}\PY{l+s+s1}{Cena biletu, a ocalenie}\PY{l+s+s1}{\PYZsq{}}\PY{p}{)}
\PY{n}{plt}\PY{o}{.}\PY{n}{xlabel}\PY{p}{(}\PY{l+s+s1}{\PYZsq{}}\PY{l+s+s1}{Cena biletu}\PY{l+s+s1}{\PYZsq{}}\PY{p}{)}
\PY{n}{plt}\PY{o}{.}\PY{n}{xlim}\PY{p}{(}\PY{o}{\PYZhy{}}\PY{l+m+mi}{10}\PY{p}{,} \PY{l+m+mi}{280}\PY{p}{)}
\PY{n}{plt}\PY{o}{.}\PY{n}{ylabel}\PY{p}{(}\PY{l+s+s1}{\PYZsq{}}\PY{l+s+s1}{Ocałał (0 = Nie, 1 = Tak)}\PY{l+s+s1}{\PYZsq{}}\PY{p}{)}
\PY{c+c1}{\PYZsh{}plt.xscale(\PYZsq{}log\PYZsq{})  \PYZsh{} Skala logarytmiczna dla czytelności}
\PY{n}{plt}\PY{o}{.}\PY{n}{grid}\PY{p}{(}\PY{k+kc}{True}\PY{p}{,} \PY{n}{linestyle}\PY{o}{=}\PY{l+s+s1}{\PYZsq{}}\PY{l+s+s1}{\PYZhy{}\PYZhy{}}\PY{l+s+s1}{\PYZsq{}}\PY{p}{,} \PY{n}{alpha}\PY{o}{=}\PY{l+m+mf}{0.6}\PY{p}{)}
\PY{n}{plt}\PY{o}{.}\PY{n}{show}\PY{p}{(}\PY{p}{)}
\end{Verbatim}
\end{tcolorbox}

    \begin{center}
    \adjustimage{max size={0.9\linewidth}{0.9\paperheight}}{titanic_prezentacja_files/titanic_prezentacja_108_0.png}
    \end{center}
    { \hspace*{\fill} \\}
    
    \subparagraph{Wykres wykazuje, że cena jaką pasażer zapłacił za bilet,
miała znczny wpływ na to, czy pasażer ocalał, czy
nie.}\label{wykres-wykazuje-ux17ce-cena-jakux105-pasaux17cer-zapux142aciux142-za-bilet-miaux142a-znczny-wpux142yw-na-to-czy-pasaux17cer-ocalaux142-czy-nie.}

\subparagraph{Jednak patrząc na wartości odstające mamy sporo zbliżonych
cen biletu zarówno wśród pasażerów, którzy przeżyli, jak i
zmarli.}\label{jednak-patrzux105c-na-wartoux15bci-odstajux105ce-mamy-sporo-zbliux17conych-cen-biletu-zaruxf3wno-wux15bruxf3d-pasaux17ceruxf3w-ktuxf3rzy-przeux17cyli-jak-i-zmarli.}

    \paragraph{Ocaleni, płeć, klasa.}\label{ocaleni-pux142eux107-klasa.}

    \begin{tcolorbox}[breakable, size=fbox, boxrule=1pt, pad at break*=1mm,colback=cellbackground, colframe=cellborder]
\prompt{In}{incolor}{112}{\boxspacing}
\begin{Verbatim}[commandchars=\\\{\}]
\PY{c+c1}{\PYZsh{} Tworzenie wykresu mozaikowego}
\PY{n}{plt}\PY{o}{.}\PY{n}{figure}\PY{p}{(}\PY{n}{figsize}\PY{o}{=}\PY{p}{(}\PY{l+m+mi}{10}\PY{p}{,} \PY{l+m+mi}{6}\PY{p}{)}\PY{p}{)}
\PY{n}{mosaic}\PY{p}{(}\PY{n}{df}\PY{p}{,} \PY{p}{[}\PY{l+s+s1}{\PYZsq{}}\PY{l+s+s1}{pclass}\PY{l+s+s1}{\PYZsq{}}\PY{p}{,} \PY{l+s+s1}{\PYZsq{}}\PY{l+s+s1}{sex}\PY{l+s+s1}{\PYZsq{}}\PY{p}{,} \PY{l+s+s1}{\PYZsq{}}\PY{l+s+s1}{survived}\PY{l+s+s1}{\PYZsq{}}\PY{p}{]}\PY{p}{,} \PY{n}{title}\PY{o}{=}\PY{l+s+s1}{\PYZsq{}}\PY{l+s+s1}{Przeżywalność według płci i klasy pasażerskiej}\PY{l+s+s1}{\PYZsq{}}\PY{p}{,} \PY{n}{gap}\PY{o}{=}\PY{l+m+mf}{0.02}\PY{p}{)}
\PY{n}{plt}\PY{o}{.}\PY{n}{show}\PY{p}{(}\PY{p}{)}
\end{Verbatim}
\end{tcolorbox}

    
    \begin{Verbatim}[commandchars=\\\{\}]
<Figure size 1000x600 with 0 Axes>
    \end{Verbatim}

    
    \begin{center}
    \adjustimage{max size={0.9\linewidth}{0.9\paperheight}}{titanic_prezentacja_files/titanic_prezentacja_111_1.png}
    \end{center}
    { \hspace*{\fill} \\}
    
    \subparagraph{Kobiety przeważają pod względem ocalenia. Im wyższa klasa
pasażerska, tym większy odsetek kobier ocalał. Wśród mężczyzn największy
odsetek ocalałych jest w klasie
1.}\label{kobiety-przewaux17cajux105-pod-wzglux119dem-ocalenia.-im-wyux17csza-klasa-pasaux17cerska-tym-wiux119kszy-odsetek-kobier-ocalaux142.-wux15bruxf3d-mux119ux17cczyzn-najwiux119kszy-odsetek-ocalaux142ych-jest-w-klasie-1.}

    \paragraph{Łodzie ratunkowe, klasa
pasażerska.}\label{ux142odzie-ratunkowe-klasa-pasaux17cerska.}

    \begin{tcolorbox}[breakable, size=fbox, boxrule=1pt, pad at break*=1mm,colback=cellbackground, colframe=cellborder]
\prompt{In}{incolor}{113}{\boxspacing}
\begin{Verbatim}[commandchars=\\\{\}]
\PY{n}{pd}\PY{o}{.}\PY{n}{DataFrame}\PY{p}{(}\PY{n}{df}\PY{o}{.}\PY{n}{groupby}\PY{p}{(}\PY{l+s+s1}{\PYZsq{}}\PY{l+s+s1}{pclass}\PY{l+s+s1}{\PYZsq{}}\PY{p}{)}\PY{p}{[}\PY{l+s+s1}{\PYZsq{}}\PY{l+s+s1}{boat}\PY{l+s+s1}{\PYZsq{}}\PY{p}{]}\PY{o}{.}\PY{n}{count}\PY{p}{(}\PY{p}{)}\PY{p}{)}
\end{Verbatim}
\end{tcolorbox}

            \begin{tcolorbox}[breakable, size=fbox, boxrule=.5pt, pad at break*=1mm, opacityfill=0]
\prompt{Out}{outcolor}{113}{\boxspacing}
\begin{Verbatim}[commandchars=\\\{\}]
        boat
pclass
1.0      201
2.0      112
3.0      173
\end{Verbatim}
\end{tcolorbox}
        
    \subparagraph{Na łodziach ratunkowych zarejestrowano 201 osób z klasy 1,
112osób z klasy 2 oraz 173 osoby z klasy
3.}\label{na-ux142odziach-ratunkowych-zarejestrowano-201-osuxf3b-z-klasy-1-112osuxf3b-z-klasy-2-oraz-173-osoby-z-klasy-3.}

    \paragraph{Ciała ofiar z podziałem na
klase.}\label{ciaux142a-ofiar-z-podziaux142em-na-klase.}

    \begin{tcolorbox}[breakable, size=fbox, boxrule=1pt, pad at break*=1mm,colback=cellbackground, colframe=cellborder]
\prompt{In}{incolor}{114}{\boxspacing}
\begin{Verbatim}[commandchars=\\\{\}]
\PY{n}{pd}\PY{o}{.}\PY{n}{DataFrame}\PY{p}{(}\PY{n}{df}\PY{o}{.}\PY{n}{groupby}\PY{p}{(}\PY{l+s+s1}{\PYZsq{}}\PY{l+s+s1}{pclass}\PY{l+s+s1}{\PYZsq{}}\PY{p}{)}\PY{p}{[}\PY{l+s+s1}{\PYZsq{}}\PY{l+s+s1}{body}\PY{l+s+s1}{\PYZsq{}}\PY{p}{]}\PY{o}{.}\PY{n}{count}\PY{p}{(}\PY{p}{)}\PY{p}{)}
\end{Verbatim}
\end{tcolorbox}

            \begin{tcolorbox}[breakable, size=fbox, boxrule=.5pt, pad at break*=1mm, opacityfill=0]
\prompt{Out}{outcolor}{114}{\boxspacing}
\begin{Verbatim}[commandchars=\\\{\}]
        body
pclass
1.0       35
2.0       31
3.0       55
\end{Verbatim}
\end{tcolorbox}
        
    \subparagraph{Odnaleziono 35 ciał spośród ofiar z 1 klasy, 31 ciał
spośród ofiar z 2 klasy, 55 ciał spośród ofiar z 3
klasy.}\label{odnaleziono-35-ciaux142-spoux15bruxf3d-ofiar-z-1-klasy-31-ciaux142-spoux15bruxf3d-ofiar-z-2-klasy-55-ciaux142-spoux15bruxf3d-ofiar-z-3-klasy.}

    \paragraph{Macierz korelacji dla kolumn
numerycznych.}\label{macierz-korelacji-dla-kolumn-numerycznych.}

    \begin{tcolorbox}[breakable, size=fbox, boxrule=1pt, pad at break*=1mm,colback=cellbackground, colframe=cellborder]
\prompt{In}{incolor}{115}{\boxspacing}
\begin{Verbatim}[commandchars=\\\{\}]
\PY{c+c1}{\PYZsh{} Obliczenie macierzy korelacji}
\PY{n}{correlation\PYZus{}matrix} \PY{o}{=} \PY{n}{df}\PY{p}{[}\PY{p}{[}\PY{l+s+s1}{\PYZsq{}}\PY{l+s+s1}{age}\PY{l+s+s1}{\PYZsq{}}\PY{p}{,}\PY{l+s+s1}{\PYZsq{}}\PY{l+s+s1}{sibsp}\PY{l+s+s1}{\PYZsq{}}\PY{p}{,}\PY{l+s+s1}{\PYZsq{}}\PY{l+s+s1}{parch}\PY{l+s+s1}{\PYZsq{}}\PY{p}{,} \PY{l+s+s1}{\PYZsq{}}\PY{l+s+s1}{fare}\PY{l+s+s1}{\PYZsq{}}\PY{p}{,} \PY{l+s+s1}{\PYZsq{}}\PY{l+s+s1}{body}\PY{l+s+s1}{\PYZsq{}}\PY{p}{,} \PY{l+s+s1}{\PYZsq{}}\PY{l+s+s1}{survived}\PY{l+s+s1}{\PYZsq{}}\PY{p}{,} \PY{l+s+s1}{\PYZsq{}}\PY{l+s+s1}{pclass}\PY{l+s+s1}{\PYZsq{}}\PY{p}{]}\PY{p}{]}\PY{o}{.}\PY{n}{corr}\PY{p}{(}\PY{p}{)}

\PY{c+c1}{\PYZsh{} Wizualizacja macierzy korelacji}
\PY{n}{plt}\PY{o}{.}\PY{n}{figure}\PY{p}{(}\PY{n}{figsize}\PY{o}{=}\PY{p}{(}\PY{l+m+mi}{10}\PY{p}{,} \PY{l+m+mi}{6}\PY{p}{)}\PY{p}{)}
\PY{n}{sns}\PY{o}{.}\PY{n}{heatmap}\PY{p}{(}\PY{n}{correlation\PYZus{}matrix}\PY{p}{,} \PY{n}{annot}\PY{o}{=}\PY{k+kc}{True}\PY{p}{,} \PY{n}{cmap}\PY{o}{=}\PY{l+s+s2}{\PYZdq{}}\PY{l+s+s2}{coolwarm}\PY{l+s+s2}{\PYZdq{}}\PY{p}{,} \PY{n}{fmt}\PY{o}{=}\PY{l+s+s2}{\PYZdq{}}\PY{l+s+s2}{.2f}\PY{l+s+s2}{\PYZdq{}}\PY{p}{,} \PY{n}{linewidths}\PY{o}{=}\PY{l+m+mf}{0.5}\PY{p}{)}
\PY{n}{plt}\PY{o}{.}\PY{n}{title}\PY{p}{(}\PY{l+s+s2}{\PYZdq{}}\PY{l+s+s2}{Macierz korelacji dla danych Titanica}\PY{l+s+s2}{\PYZdq{}}\PY{p}{)}
\PY{n}{plt}\PY{o}{.}\PY{n}{show}\PY{p}{(}\PY{p}{)}
\end{Verbatim}
\end{tcolorbox}

    \begin{center}
    \adjustimage{max size={0.9\linewidth}{0.9\paperheight}}{titanic_prezentacja_files/titanic_prezentacja_120_0.png}
    \end{center}
    { \hspace*{\fill} \\}
    
    \subparagraph{Widzimy korelację pomiędzy rodzinami(sibsp i parch). A
także odwróconą korelacje pomiedzy klasą pasażerską(pclass), a
wiekiem(age), ceną biletu(fare) i
ocalonymi(survived).}\label{widzimy-korelacjux119-pomiux119dzy-rodzinamisibsp-i-parch.-a-takux17ce-odwruxf3conux105-korelacje-pomiedzy-klasux105-pasaux17cerskux105pclass-a-wiekiemage-cenux105-biletufare-i-ocalonymisurvived.}

    \section{6. Wartości odstające.}\label{wartoux15bci-odstajux105ce.}

    \paragraph{Cena biletu.}\label{cena-biletu.}

    \begin{tcolorbox}[breakable, size=fbox, boxrule=1pt, pad at break*=1mm,colback=cellbackground, colframe=cellborder]
\prompt{In}{incolor}{116}{\boxspacing}
\begin{Verbatim}[commandchars=\\\{\}]
\PY{n}{fare\PYZus{}stats} \PY{o}{=} \PY{n}{df}\PY{o}{.}\PY{n}{groupby}\PY{p}{(}\PY{p}{[}\PY{l+s+s1}{\PYZsq{}}\PY{l+s+s1}{pclass}\PY{l+s+s1}{\PYZsq{}}\PY{p}{]}\PY{p}{)}\PY{p}{[}\PY{l+s+s1}{\PYZsq{}}\PY{l+s+s1}{fare}\PY{l+s+s1}{\PYZsq{}}\PY{p}{]}\PY{o}{.}\PY{n}{describe}\PY{p}{(}\PY{p}{)}
\PY{n}{fare\PYZus{}stats}
\end{Verbatim}
\end{tcolorbox}

            \begin{tcolorbox}[breakable, size=fbox, boxrule=.5pt, pad at break*=1mm, opacityfill=0]
\prompt{Out}{outcolor}{116}{\boxspacing}
\begin{Verbatim}[commandchars=\\\{\}]
        count       mean        std  min      25\%      50\%       75\%       max
pclass
1.0     323.0  87.508992  80.447178  0.0  30.6958  60.0000  107.6625  512.3292
2.0     277.0  21.179196  13.607122  0.0  13.0000  15.0458   26.0000   73.5000
3.0     709.0  13.302885  11.486238  0.0   7.7500   8.0500   15.2458   69.5500
\end{Verbatim}
\end{tcolorbox}
        
    \begin{tcolorbox}[breakable, size=fbox, boxrule=1pt, pad at break*=1mm,colback=cellbackground, colframe=cellborder]
\prompt{In}{incolor}{117}{\boxspacing}
\begin{Verbatim}[commandchars=\\\{\}]
\PY{n}{df\PYZus{}temp}\PY{p}{[}\PY{l+s+s1}{\PYZsq{}}\PY{l+s+s1}{pclass}\PY{l+s+s1}{\PYZsq{}}\PY{p}{]} \PY{o}{=} \PY{n}{df\PYZus{}temp}\PY{p}{[}\PY{l+s+s1}{\PYZsq{}}\PY{l+s+s1}{pclass}\PY{l+s+s1}{\PYZsq{}}\PY{p}{]}\PY{o}{.}\PY{n}{astype}\PY{p}{(}\PY{n+nb}{str}\PY{p}{)}

\PY{n}{plt}\PY{o}{.}\PY{n}{figure}\PY{p}{(}\PY{n}{figsize}\PY{o}{=}\PY{p}{(}\PY{l+m+mi}{12}\PY{p}{,} \PY{l+m+mi}{5}\PY{p}{)}\PY{p}{)}
\PY{n}{sns}\PY{o}{.}\PY{n}{boxplot}\PY{p}{(}\PY{n}{x}\PY{o}{=}\PY{l+s+s1}{\PYZsq{}}\PY{l+s+s1}{fare}\PY{l+s+s1}{\PYZsq{}}\PY{p}{,} \PY{n}{y}\PY{o}{=}\PY{l+s+s1}{\PYZsq{}}\PY{l+s+s1}{pclass}\PY{l+s+s1}{\PYZsq{}}\PY{p}{,} \PY{n}{data}\PY{o}{=}\PY{n}{df\PYZus{}temp}\PY{p}{,} \PY{n}{hue}\PY{o}{=}\PY{l+s+s1}{\PYZsq{}}\PY{l+s+s1}{pclass}\PY{l+s+s1}{\PYZsq{}}\PY{p}{,} \PY{n}{vert}\PY{o}{=}\PY{k+kc}{False}\PY{p}{,}\PY{p}{)}

\PY{n}{plt}\PY{o}{.}\PY{n}{title}\PY{p}{(}\PY{l+s+s1}{\PYZsq{}}\PY{l+s+s1}{Cena biletu a klasa pasażerska}\PY{l+s+s1}{\PYZsq{}}\PY{p}{)}
\PY{n}{plt}\PY{o}{.}\PY{n}{xlabel}\PY{p}{(}\PY{l+s+s1}{\PYZsq{}}\PY{l+s+s1}{Cena biletu}\PY{l+s+s1}{\PYZsq{}}\PY{p}{)}
\PY{n}{plt}\PY{o}{.}\PY{n}{xlim}\PY{p}{(}\PY{o}{\PYZhy{}}\PY{l+m+mi}{10}\PY{p}{,} \PY{l+m+mi}{550}\PY{p}{)}
\PY{n}{plt}\PY{o}{.}\PY{n}{ylabel}\PY{p}{(}\PY{l+s+s1}{\PYZsq{}}\PY{l+s+s1}{Klasa pasażerska}\PY{l+s+s1}{\PYZsq{}}\PY{p}{)}
\PY{c+c1}{\PYZsh{}plt.xscale(\PYZsq{}log\PYZsq{})  \PYZsh{} Skala logarytmiczna dla czytelności}
\PY{n}{plt}\PY{o}{.}\PY{n}{grid}\PY{p}{(}\PY{k+kc}{True}\PY{p}{,} \PY{n}{linestyle}\PY{o}{=}\PY{l+s+s1}{\PYZsq{}}\PY{l+s+s1}{\PYZhy{}\PYZhy{}}\PY{l+s+s1}{\PYZsq{}}\PY{p}{,} \PY{n}{alpha}\PY{o}{=}\PY{l+m+mf}{0.8}\PY{p}{)}
\PY{n}{plt}\PY{o}{.}\PY{n}{show}\PY{p}{(}\PY{p}{)}
\end{Verbatim}
\end{tcolorbox}

    \begin{center}
    \adjustimage{max size={0.9\linewidth}{0.9\paperheight}}{titanic_prezentacja_files/titanic_prezentacja_125_0.png}
    \end{center}
    { \hspace*{\fill} \\}
    
    \subparagraph{Wartości odstające dla cen biletów największą rozpiętość
mają w klasie 1: w przybliżeniu od 220 do 520, w klasie 2 od 40 do 75, w
klasie 3 od 25 do
70}\label{wartoux15bci-odstajux105ce-dla-cen-biletuxf3w-najwiux119kszux105-rozpiux119toux15bux107-majux105-w-klasie-1-w-przybliux17ceniu-od-220-do-520-w-klasie-2-od-40-do-75-w-klasie-3-od-25-do-70}

    \paragraph{Wiek.}\label{wiek.}

    \begin{tcolorbox}[breakable, size=fbox, boxrule=1pt, pad at break*=1mm,colback=cellbackground, colframe=cellborder]
\prompt{In}{incolor}{118}{\boxspacing}
\begin{Verbatim}[commandchars=\\\{\}]
\PY{n}{pd}\PY{o}{.}\PY{n}{DataFrame}\PY{p}{(}\PY{n}{df}\PY{p}{[}\PY{l+s+s1}{\PYZsq{}}\PY{l+s+s1}{age}\PY{l+s+s1}{\PYZsq{}}\PY{p}{]}\PY{o}{.}\PY{n}{describe}\PY{p}{(}\PY{p}{)}\PY{p}{)}
\end{Verbatim}
\end{tcolorbox}

            \begin{tcolorbox}[breakable, size=fbox, boxrule=.5pt, pad at break*=1mm, opacityfill=0]
\prompt{Out}{outcolor}{118}{\boxspacing}
\begin{Verbatim}[commandchars=\\\{\}]
               age
count  1309.000000
mean     29.978610
std      12.889776
min       0.000000
25\%      22.000000
50\%      30.000000
75\%      35.000000
max      80.000000
\end{Verbatim}
\end{tcolorbox}
        
    \begin{tcolorbox}[breakable, size=fbox, boxrule=1pt, pad at break*=1mm,colback=cellbackground, colframe=cellborder]
\prompt{In}{incolor}{119}{\boxspacing}
\begin{Verbatim}[commandchars=\\\{\}]
\PY{c+c1}{\PYZsh{} Tworzenie wykresu pudełkowego (boxplot)}
\PY{n}{plt}\PY{o}{.}\PY{n}{figure}\PY{p}{(}\PY{n}{figsize}\PY{o}{=}\PY{p}{(}\PY{l+m+mi}{10}\PY{p}{,} \PY{l+m+mi}{3}\PY{p}{)}\PY{p}{)}
\PY{n}{sns}\PY{o}{.}\PY{n}{boxplot}\PY{p}{(}\PY{n}{x}\PY{o}{=}\PY{n}{df}\PY{p}{[}\PY{l+s+s1}{\PYZsq{}}\PY{l+s+s1}{age}\PY{l+s+s1}{\PYZsq{}}\PY{p}{]}\PY{p}{,} \PY{n}{color}\PY{o}{=}\PY{l+s+s1}{\PYZsq{}}\PY{l+s+s1}{lightblue}\PY{l+s+s1}{\PYZsq{}}\PY{p}{,} \PY{n}{vert}\PY{o}{=}\PY{k+kc}{False}\PY{p}{)}

\PY{c+c1}{\PYZsh{} Opisy wykresu}
\PY{n}{plt}\PY{o}{.}\PY{n}{title}\PY{p}{(}\PY{l+s+s1}{\PYZsq{}}\PY{l+s+s1}{Boxplot dla wieku pasażerów Titanica}\PY{l+s+s1}{\PYZsq{}}\PY{p}{)}
\PY{n}{plt}\PY{o}{.}\PY{n}{grid}\PY{p}{(}\PY{k+kc}{True}\PY{p}{,} \PY{n}{linestyle}\PY{o}{=}\PY{l+s+s1}{\PYZsq{}}\PY{l+s+s1}{\PYZhy{}\PYZhy{}}\PY{l+s+s1}{\PYZsq{}}\PY{p}{,} \PY{n}{alpha}\PY{o}{=}\PY{l+m+mf}{0.6}\PY{p}{)}

\PY{n}{plt}\PY{o}{.}\PY{n}{show}\PY{p}{(}\PY{p}{)}
\end{Verbatim}
\end{tcolorbox}

    \begin{center}
    \adjustimage{max size={0.9\linewidth}{0.9\paperheight}}{titanic_prezentacja_files/titanic_prezentacja_129_0.png}
    \end{center}
    { \hspace*{\fill} \\}
    
    \subparagraph{Dane o wieku posiadają wartości odstające zarówno przy
wartościach minimalnych - poniżej 2 lat, jak i maksymalnych - powyżej 53
lata.}\label{dane-o-wieku-posiadajux105-wartoux15bci-odstajux105ce-zaruxf3wno-przy-wartoux15bciach-minimalnych---poniux17cej-2-lat-jak-i-maksymalnych---powyux17cej-53-lata.}

    \paragraph{Liczba rodzeństwa, małżonków na
pokładzie}\label{liczba-rodzeux144stwa-maux142ux17conkuxf3w-na-pokux142adzie}

    \begin{tcolorbox}[breakable, size=fbox, boxrule=1pt, pad at break*=1mm,colback=cellbackground, colframe=cellborder]
\prompt{In}{incolor}{120}{\boxspacing}
\begin{Verbatim}[commandchars=\\\{\}]
\PY{n}{pd}\PY{o}{.}\PY{n}{DataFrame}\PY{p}{(}\PY{n}{df}\PY{p}{[}\PY{l+s+s1}{\PYZsq{}}\PY{l+s+s1}{sibsp}\PY{l+s+s1}{\PYZsq{}}\PY{p}{]}\PY{o}{.}\PY{n}{describe}\PY{p}{(}\PY{p}{)}\PY{p}{)}
\end{Verbatim}
\end{tcolorbox}

            \begin{tcolorbox}[breakable, size=fbox, boxrule=.5pt, pad at break*=1mm, opacityfill=0]
\prompt{Out}{outcolor}{120}{\boxspacing}
\begin{Verbatim}[commandchars=\\\{\}]
             sibsp
count  1309.000000
mean      0.498854
std       1.041658
min       0.000000
25\%       0.000000
50\%       0.000000
75\%       1.000000
max       8.000000
\end{Verbatim}
\end{tcolorbox}
        
    \begin{tcolorbox}[breakable, size=fbox, boxrule=1pt, pad at break*=1mm,colback=cellbackground, colframe=cellborder]
\prompt{In}{incolor}{121}{\boxspacing}
\begin{Verbatim}[commandchars=\\\{\}]
\PY{c+c1}{\PYZsh{} Tworzenie wykresu pudełkowego (boxplot)}
\PY{n}{plt}\PY{o}{.}\PY{n}{figure}\PY{p}{(}\PY{n}{figsize}\PY{o}{=}\PY{p}{(}\PY{l+m+mi}{10}\PY{p}{,} \PY{l+m+mi}{3}\PY{p}{)}\PY{p}{)}
\PY{n}{sns}\PY{o}{.}\PY{n}{boxplot}\PY{p}{(}\PY{n}{x}\PY{o}{=}\PY{n}{df}\PY{p}{[}\PY{l+s+s1}{\PYZsq{}}\PY{l+s+s1}{sibsp}\PY{l+s+s1}{\PYZsq{}}\PY{p}{]}\PY{p}{,} \PY{n}{color}\PY{o}{=}\PY{l+s+s1}{\PYZsq{}}\PY{l+s+s1}{lawngreen}\PY{l+s+s1}{\PYZsq{}}\PY{p}{,} \PY{n}{vert}\PY{o}{=}\PY{k+kc}{False}\PY{p}{)}

\PY{c+c1}{\PYZsh{} Opisy wykresu}
\PY{n}{plt}\PY{o}{.}\PY{n}{title}\PY{p}{(}\PY{l+s+s1}{\PYZsq{}}\PY{l+s+s1}{Boxplot dla  liczby rodzeństwa, małżonków na pokładzie}\PY{l+s+s1}{\PYZsq{}}\PY{p}{)}
\PY{n}{plt}\PY{o}{.}\PY{n}{xlim}\PY{p}{(}\PY{o}{\PYZhy{}}\PY{l+m+mi}{1}\PY{p}{,} \PY{l+m+mi}{10}\PY{p}{)}
\PY{n}{plt}\PY{o}{.}\PY{n}{grid}\PY{p}{(}\PY{k+kc}{True}\PY{p}{,} \PY{n}{linestyle}\PY{o}{=}\PY{l+s+s1}{\PYZsq{}}\PY{l+s+s1}{\PYZhy{}\PYZhy{}}\PY{l+s+s1}{\PYZsq{}}\PY{p}{,} \PY{n}{alpha}\PY{o}{=}\PY{l+m+mf}{0.6}\PY{p}{)}

\PY{n}{plt}\PY{o}{.}\PY{n}{show}\PY{p}{(}\PY{p}{)}
\end{Verbatim}
\end{tcolorbox}

    \begin{center}
    \adjustimage{max size={0.9\linewidth}{0.9\paperheight}}{titanic_prezentacja_files/titanic_prezentacja_133_0.png}
    \end{center}
    { \hspace*{\fill} \\}
    
    \subparagraph{75\% onserwacji miało 1 członka rodziny na pokładzie,
wartości odstające było sięgały 8 członków
rodziny.}\label{onserwacji-miaux142o-1-czux142onka-rodziny-na-pokux142adzie-wartoux15bci-odstajux105ce-byux142o-siux119gaux142y-8-czux142onkuxf3w-rodziny.}

    \paragraph{Liczba rodziców, dzieci na
pokładzie.}\label{liczba-rodzicuxf3w-dzieci-na-pokux142adzie.}

    \begin{tcolorbox}[breakable, size=fbox, boxrule=1pt, pad at break*=1mm,colback=cellbackground, colframe=cellborder]
\prompt{In}{incolor}{122}{\boxspacing}
\begin{Verbatim}[commandchars=\\\{\}]
\PY{n}{pd}\PY{o}{.}\PY{n}{DataFrame}\PY{p}{(}\PY{n}{df}\PY{p}{[}\PY{l+s+s1}{\PYZsq{}}\PY{l+s+s1}{parch}\PY{l+s+s1}{\PYZsq{}}\PY{p}{]}\PY{o}{.}\PY{n}{describe}\PY{p}{(}\PY{p}{)}\PY{o}{.}\PY{n}{dropna}\PY{p}{(}\PY{p}{)}\PY{p}{)}
\end{Verbatim}
\end{tcolorbox}

            \begin{tcolorbox}[breakable, size=fbox, boxrule=.5pt, pad at break*=1mm, opacityfill=0]
\prompt{Out}{outcolor}{122}{\boxspacing}
\begin{Verbatim}[commandchars=\\\{\}]
             parch
count  1309.000000
mean      0.385027
std       0.865560
min       0.000000
25\%       0.000000
50\%       0.000000
75\%       0.000000
max       9.000000
\end{Verbatim}
\end{tcolorbox}
        
    \begin{tcolorbox}[breakable, size=fbox, boxrule=1pt, pad at break*=1mm,colback=cellbackground, colframe=cellborder]
\prompt{In}{incolor}{123}{\boxspacing}
\begin{Verbatim}[commandchars=\\\{\}]
\PY{c+c1}{\PYZsh{} Tworzenie wykresu pudełkowego (boxplot)}
\PY{n}{plt}\PY{o}{.}\PY{n}{figure}\PY{p}{(}\PY{n}{figsize}\PY{o}{=}\PY{p}{(}\PY{l+m+mi}{10}\PY{p}{,} \PY{l+m+mi}{3}\PY{p}{)}\PY{p}{)}
\PY{n}{sns}\PY{o}{.}\PY{n}{boxplot}\PY{p}{(}\PY{n}{x}\PY{o}{=}\PY{n}{df}\PY{p}{[}\PY{l+s+s1}{\PYZsq{}}\PY{l+s+s1}{parch}\PY{l+s+s1}{\PYZsq{}}\PY{p}{]}\PY{p}{,} \PY{n}{color}\PY{o}{=}\PY{l+s+s1}{\PYZsq{}}\PY{l+s+s1}{orangered}\PY{l+s+s1}{\PYZsq{}}\PY{p}{,} \PY{n}{vert}\PY{o}{=}\PY{k+kc}{False}\PY{p}{)}

\PY{c+c1}{\PYZsh{} Opisy wykresu}
\PY{n}{plt}\PY{o}{.}\PY{n}{title}\PY{p}{(}\PY{l+s+s1}{\PYZsq{}}\PY{l+s+s1}{Boxplot dla  liczby rodziców, dzieci na pokładzie}\PY{l+s+s1}{\PYZsq{}}\PY{p}{)}
\PY{n}{plt}\PY{o}{.}\PY{n}{xlim}\PY{p}{(}\PY{o}{\PYZhy{}}\PY{l+m+mi}{1}\PY{p}{,} \PY{l+m+mi}{10}\PY{p}{)}
\PY{n}{plt}\PY{o}{.}\PY{n}{grid}\PY{p}{(}\PY{k+kc}{True}\PY{p}{,} \PY{n}{linestyle}\PY{o}{=}\PY{l+s+s1}{\PYZsq{}}\PY{l+s+s1}{\PYZhy{}\PYZhy{}}\PY{l+s+s1}{\PYZsq{}}\PY{p}{,} \PY{n}{alpha}\PY{o}{=}\PY{l+m+mf}{0.6}\PY{p}{)}

\PY{n}{plt}\PY{o}{.}\PY{n}{show}\PY{p}{(}\PY{p}{)}
\end{Verbatim}
\end{tcolorbox}

    \begin{center}
    \adjustimage{max size={0.9\linewidth}{0.9\paperheight}}{titanic_prezentacja_files/titanic_prezentacja_137_0.png}
    \end{center}
    { \hspace*{\fill} \\}
    
    \paragraph{75\% obserwacji nie miało żadnego członka rodziny na
pokładzie. Wartości odstające sięgały 9 członków
rodziny.}\label{obserwacji-nie-miaux142o-ux17cadnego-czux142onka-rodziny-na-pokux142adzie.-wartoux15bci-odstajux105ce-siux119gaux142y-9-czux142onkuxf3w-rodziny.}

    \section{Podsumowanie}\label{podsumowanie}

\paragraph{Liczebność i zakres
danych:}\label{liczebnoux15bux107-i-zakres-danych}

Analizowany zbiór obejmuje 1310 pasażerów i 14 atrybutów, takich jak
klasa podróży, wiek, płeć, liczba członków rodziny na pokładzie, cena
biletu, miejsce zaokrętowania, numer kabiny, łodzi ratunkowej, ciała
oraz cel podróży.

\paragraph{Przeżywalność:}\label{przeux17cywalnoux15bux107}

Katastrofę przeżyło 38\% pasażerów (500 osób), z czego zdecydowaną
większość stanowiły kobiety (339 kobiet vs.~161 mężczyzn).

\paragraph{Klasa podróży:}\label{klasa-podruxf3ux17cy}

Pasażerowie 1 klasy mieli najwyższy odsetek przeżycia (200 z 323 osób),
w 2 klasie przeżyło 119 z 277 osób, w 3 klasie -- 181 z 709 osób. Im
wyższa klasa, tym większa szansa na przeżycie.

\paragraph{Płeć:}\label{pux142eux107}

Na pokładzie było 466 kobiet i 843 mężczyzn. Kobiety miały zdecydowanie
większą szansę na przeżycie niż mężczyźni.

    \section{Podsumowanie}\label{podsumowanie}

\paragraph{Wiek:}\label{wiek}

Średni wiek pasażerów wynosił ok. 30 lat, najmłodszy pasażer miał mniej
niż rok, najstarszy 80 lat. Wiek nie miał jednoznacznego wpływu na
przeżycie, ale dzieci i kobiety były częściej ratowane.

\paragraph{Rodzina na pokładzie:}\label{rodzina-na-pokux142adzie}

49\% pasażerów podróżowało z rodzeństwem lub małżonkiem, 38\% z rodzicem
lub dzieckiem. Większe rodziny były rzadkością -- wartości odstające
sięgały 8-9 członków rodziny.

\paragraph{Cena biletu:}\label{cena-biletu}

Średnia cena biletu to 33 jednostki walutowe, przy czym w 1 klasie
średnio 87, w 2 klasie 21, w 3 klasie 13. Cena biletu silnie zależała od
klasy i była powiązana z szansą przeżycia.

\paragraph{Port zaokrętowania:}\label{port-zaokrux119towania}

Najwięcej pasażerów wsiadło w Southampton (914), następnie Cherbourg
(270) i Queenstown (123).

    \section{Podsumowanie}\label{podsumowanie}

\paragraph{Braki danych:}\label{braki-danych}

Najwięcej brakujących wartości dotyczyło numerów kabin (1014), wieku
(263), celu podróży (564), numerów łodzi ratunkowych (823) i ciał
(1188). Braki w wieku i cenie biletu można uzupełnić średnimi
wartościami dla płci/klasy.

\paragraph{Duplikaty:}\label{duplikaty}

Występowały powtarzające się numery biletów, ale były przypisane do
różnych osób (np. rodziny).

\paragraph{Wartości odstające:}\label{wartoux15bci-odstajux105ce}

Dotyczyły głównie cen biletów (zwłaszcza w 1 klasie) oraz liczby
członków rodziny na pokładzie.

\paragraph{Korelacje:}\label{korelacje}

Silna zależność między klasą podróży, ceną biletu a przeżyciem. Wysoka
korelacja między liczbą rodzeństwa a liczbą rodziców/dzieci na
pokładzie. Odwrócona korelacja między klasą a wiekiem, ceną biletu i
przeżyciem.

\subsubsection{Analiza potwierdza, że klasa podróży, płeć, cena biletu i
port zaokrętowania były kluczowymi czynnikami wpływającymi na szanse
przeżycia katastrofy
Titanica.}\label{analiza-potwierdza-ux17ce-klasa-podruxf3ux17cy-pux142eux107-cena-biletu-i-port-zaokrux119towania-byux142y-kluczowymi-czynnikami-wpux142ywajux105cymi-na-szanse-przeux17cycia-katastrofy-titanica.}


    % Add a bibliography block to the postdoc
    
    
    
\end{document}
